%%%%%%%%%%%%%%%%%%%%%%%%%%%%%%%%%%%%%%%%%%%%%%%%%%%%%%%%%%%%%%%%%%%%%%%%%%%%%%%
%%  
%% 文档类 NJU-Thesis 用户手册
%%
%% 作者:胡海星,starfish (at) gmail (dot) com
%% 项目主页: https://github.com/Haixing-Hu/nju-thesis
%%
%% This file may be distributed and/or modified under the conditions of the
%% LaTeX Project Public License, either version 1.2 of this license or (at your
%% option) any later version. The latest version of this license is in:
%%
%% http://www.latex-project.org/lppl.txt
%%
%% and version 1.2 or later is part of all distributions of LaTeX version
%% 1999/12/01 or later.
%%
%%%%%%%%%%%%%%%%%%%%%%%%%%%%%%%%%%%%%%%%%%%%%%%%%%%%%%%%%%%%%%%%%%%%%%%%%%%%%%%

\documentclass[phd]{njuthesis}
%% njuthesis 文档类的可选参数有:
%%   nobackinfo 取消封二页导师签名信息。注意,按照南大的规定,是需要签名页的。
%%   phd/master/bachelor 选择博士/硕士/学士论文

% 自定义的设置
%%%%%%%%%%%%%%%%%%%%%%%%%%%%%%%%%%%%%%%%%%%%%%%%%%%%%%%%%%%%%%%%%%%%%%%%%%%%%%%
%%  
%% 文档类 NJU-Thesis 用户手册
%%
%% 作者:胡海星,starfish (at) gmail (dot) com
%% 项目主页: https://github.com/Haixing-Hu/nju-thesis
%%
%% This file may be distributed and/or modified under the conditions of the
%% LaTeX Project Public License, either version 1.2 of this license or (at your
%% option) any later version. The latest version of this license is in:
%%
%% http://www.latex-project.org/lppl.txt
%%
%% and version 1.2 or later is part of all distributions of LaTeX version
%% 1999/12/01 or later.
%%
%%%%%%%%%%%%%%%%%%%%%%%%%%%%%%%%%%%%%%%%%%%%%%%%%%%%%%%%%%%%%%%%%%%%%%%%%%%%%%%

% 下面的宏包提供了\BibTeX的logo
\usepackage{dtklogos}

% 提供 framed 环境
\usepackage{framed}

% 使用\verb{text}命令的缩写形式|text|
\MakeShortVerb{\|}

% \std{xxxx}表示标准号
\newcommand{\std}[1]{\texttt{#1}}

% \dangericon 表示警告的图标
\font\manfnt=manfnt
\newcommand*{\dangericon}{\manfnt\char127}

% note 环境表示需特别注意的内容
\newenvironment{note}
{\vskip1.5ex\par\noindent\llap{\dangericon\hskip2mm}\hskip\parindent\textbf{注意:}}
{\vskip1.5ex}

% syntax 环境表示语法描述
\newenvironment{syntax}{\begin{center}}{\end{center}}
\usepackage{fancyvrb}

\DefineVerbatimEnvironment{shell}{Verbatim}%
  {frame=single,framerule=0.1mm,rulecolor=\color{black},%
   framesep=2mm,fontsize=\small}

\DefineVerbatimEnvironment{tex}{Verbatim}%
  {frame=single,framerule=0.1mm,rulecolor=\color{black},%
   framesep=2mm,baselinestretch=1.2,fontsize=\small}

\newcommand{\cs}[1]{\texttt{\textbackslash{}#1}}
\newcommand{\env}[1]{\texttt{#1}}
\newcommand{\meta}[1]{\ensuremath{\langle}\textit{#1}\ensuremath{\rangle}}
\newcommand{\oarg}[1]{\texttt{[}\meta{#1}\texttt{]}}
\newcommand{\marg}[1]{\texttt{\{}\meta{#1}\texttt{\}}}
\newcommand{\parg}[1]{\texttt{(}\meta{#1}\texttt{)}}



%%%%%%%%%%%%%%%%%%%%%%%%%%%%%%%%%%%%%%%%%%%%%%%%%%%%%%%%%%%%%%%%%%%%%%%%%%%%%%%
% 设置《国家图书馆封面》的内容,仅博士论文才需要填写

% 设置论文按照《中国图书资料分类法》的分类编号
\classification{0175.2}
% 论文的密级。需按照GB/T 7156-2003标准进行设置。预定义的值包括:
% - \openlevel,表示公开级:此级别的文献可在国内外发行和交换。
% - \controllevel,表示限制级:此级别的文献内容不涉及国家秘密,但在一定时间内
%   限制其交流和使用范围。
% - \confidentiallevel,表示秘密级:此级别的文献内容涉及一般国家秘密。
% - \clasifiedlevel,表示机密级:此级别的文献内容涉及重要的国家秘密 。
% - \mostconfidentiallevel,表示绝密级:此级别的文献内容涉及最重要的国家秘密。
% 此属性可选,默认为\openlevel,即公开级。
\securitylevel{\openlevel}
% 设置论文按照《国际十进分类法UDC》的分类编号
% 该编号可在下述网址查询:http://www.udcc.org/udcsummary/php/index.php?lang=chi
\udc{004.72}
% 国家图书馆封面上的论文标题第一行,不可换行。此属性可选,默认值为通过\title设置的标题。
\nlctitlea{文档类{\njuthesis}}
% 国家图书馆封面上的论文标题第二行,不可换行。此属性可选,默认值为空白。
\nlctitleb{用户手册}
% 国家图书馆封面上的论文标题第三行,不可换行。此属性可选,默认值为空白。
\nlctitlec{}
% 导师的单位名称及地址
\supervisorinfo{南京大学计算机科学与技术系~~南京市汉口路22号~~210093}
% 答辩委员会主席
\chairman{宋方敏~~教授}
% 第一位评阅人
\reviewera{张三~~教授}
% 第二位评阅人
\reviewerb{李四~~副教授}
% 第三位评阅人
\reviewerc{王五~~教授}
% 第四位评阅人
\reviewerd{赵六~~研究员}

%%%%%%%%%%%%%%%%%%%%%%%%%%%%%%%%%%%%%%%%%%%%%%%%%%%%%%%%%%%%%%%%%%%%%%%%%%%%%%%
% 设置论文的中文封面

% 论文标题,不可换行
\title{文档类{\njuthesis}用户手册}
% 论文作者姓名
\author{胡海星}
% 论文作者联系电话
\telphone{xxxxxxxx}
% 论文作者电子邮件地址
\email{xxxxx@gmail.com}
% 论文作者学生证号
\studentnum{xxxxxxxx}
% 论文作者入学年份(年级)
\grade{xxxx}
% 导师姓名职称
\supervisor{宋方敏~~教授}
% 导师的联系电话
\supervisortelphone{xxxxxxxxx}
% 论文作者的学科与专业方向
\major{计算机软件与理论}
% 论文作者的研究方向
\researchfield{量子算法与量子逻辑}
% 论文作者所在院系的中文名称
\department{计算机科学与技术系}
% 论文作者所在学校或机构的名称。此属性可选,默认值为``南京大学''。
\institute{南京大学}
% 论文的提交日期,需设置年、月、日。
\submitdate{2013年9月5日}
% 论文的答辩日期,需设置年、月、日。
\defenddate{2013年9月20日}
% 论文的定稿日期,需设置年、月、日。此属性可选,默认值为最后一次编译时的日期,精确到日。
\date{2013年8月28日}

%%%%%%%%%%%%%%%%%%%%%%%%%%%%%%%%%%%%%%%%%%%%%%%%%%%%%%%%%%%%%%%%%%%%%%%%%%%%%%%
% 设置论文的英文封面

% 论文的英文标题,注意其中不可换行
\englishtitle{The User Manual of the {\njuthesis} Document Class}
% 论文作者姓名的拼音
\englishauthor{HU Hai-Xing}
% 导师姓名职称的英文
\englishsupervisor{Professor SONG Fang-Min}
% 论文作者学科与专业的英文名
\englishmajor{Computer Software and Theory}
% 论文作者所在院系的英文名称
\englishdepartment{Department of Computer Science and Technology}
% 论文作者所在学校或机构的英文名称。此属性可选,默认值为``Nanjing University''。
\englishinstitute{Nanjing University}
% 论文完成日期的英文形式,它将出现在英文封面下方。需设置年、月、日。日期格式使用美国的日期
% 格式,即``Month day, year'',其中``Month''为月份的英文名全称,首字母大写;``day''为
% 该月中日期的阿拉伯数字表示;``year''为年份的四位阿拉伯数字表示。此属性可选,默认值为最后
% 一次编译时的日期。
\englishdate{Aug 28, 2013}

%%%%%%%%%%%%%%%%%%%%%%%%%%%%%%%%%%%%%%%%%%%%%%%%%%%%%%%%%%%%%%%%%%%%%%%%%%%%%%%
\begin{document}

% 制作国家图书馆封面(博士学位论文才需要)
\makenlctitle
% 制作中文封面
\maketitle
% 制作英文封面
\makeenglishtitle
% 开始前言部分
\frontmatter
% 论文的中文摘要
%%%%%%%%%%%%%%%%%%%%%%%%%%%%%%%%%%%%%%%%%%%%%%%%%%%%%%%%%%%%%%%%%%%%%%%%%%%%%%%
%%  
%% 文档类 NJU-Thesis 用户手册
%%
%% 作者:胡海星,starfish (at) gmail (dot) com
%% 项目主页: https://github.com/Haixing-Hu/nju-thesis
%%
%% This file may be distributed and/or modified under the conditions of the
%% LaTeX Project Public License, either version 1.2 of this license or (at your
%% option) any later version. The latest version of this license is in:
%%
%% http://www.latex-project.org/lppl.txt
%%
%% and version 1.2 or later is part of all distributions of LaTeX version
%% 1999/12/01 or later.
%%
%%%%%%%%%%%%%%%%%%%%%%%%%%%%%%%%%%%%%%%%%%%%%%%%%%%%%%%%%%%%%%%%%%%%%%%%%%%%%%%


% 设置论文的中文摘要
% 设置中文摘要页面的论文标题及副标题的第一行。
% 此属性可选,其默认值为使用|\title|命令所设置的论文标题
%% \abstracttitlea{}
% 设置中文摘要页面的论文标题及副标题的第二行。
% 此属性可选,其默认值为空白
%% \abstracttitleb{}

% 论文的中文摘要
\begin{abstract}

本文档是南京大学学位论文{\XeLaTeX}文档类{\njuthesis}的用户手册。本文档详细描述了
{\njuthesis}文档类的安装和使用方法。本文档本身也使用了{\njuthesis}文档类进行排版,
因此本文档的源码可作为{\njuthesis}的使用样例和起始模板。
% 中文关键词。关键词之间用中文全角分号隔开,末尾无标点符号。
\keywords{学位论文;排版;{\XeLaTeX};南京大学}
\end{abstract}

% 论文的英文摘要
%%%%%%%%%%%%%%%%%%%%%%%%%%%%%%%%%%%%%%%%%%%%%%%%%%%%%%%%%%%%%%%%%%%%%%%%%%%%%%%
%%  
%% 文档类 NJU-Thesis 用户手册
%%
%% 作者:胡海星,starfish (at) gmail (dot) com
%% 项目主页: https://github.com/Haixing-Hu/nju-thesis
%%
%% This file may be distributed and/or modified under the conditions of the
%% LaTeX Project Public License, either version 1.2 of this license or (at your
%% option) any later version. The latest version of this license is in:
%%
%% http://www.latex-project.org/lppl.txt
%%
%% and version 1.2 or later is part of all distributions of LaTeX version
%% 1999/12/01 or later.
%%
%%%%%%%%%%%%%%%%%%%%%%%%%%%%%%%%%%%%%%%%%%%%%%%%%%%%%%%%%%%%%%%%%%%%%%%%%%%%%%%


% 设置论文的英文摘要
% 设置英文摘要页面的论文标题及副标题的第一行。
% 此属性可选,其默认值为使用|\englishtitle|命令所设置的论文标题
%% \englishabstracttitlea{}
% 设置英文摘要页面的论文标题及副标题的第二行。
% 此属性可选,其默认值为空白
%% \englishabstracttitleb{}
% 论文的英文摘要
\begin{englishabstract}
This document is the user manual of the {\XeLaTeX} document class for
typesetting the degree thesis of Nanjing University. The document describes the
installation and usage of the {\njuthesis} document class in details. Since the
document itself is typeset by the {\njuthesis} document class, it could also be
used as a usage example and starting template for the document class.

% 英文关键词。关键词之间用英文半角逗号隔开,末尾无符号。
\englishkeywords{degree thesis, typesetting, {\XeLaTeX}, Nanjing Univeristy}
\end{englishabstract}

% 论文的前言,应放在目录之前,中英文摘要之后
%%%%%%%%%%%%%%%%%%%%%%%%%%%%%%%%%%%%%%%%%%%%%%%%%%%%%%%%%%%%%%%%%%%%%%%%%%%%%%%
%%  
%% 文档类 NJU-Thesis 用户手册
%%
%% 作者:胡海星,starfish (at) gmail (dot) com
%% 项目主页: https://github.com/Haixing-Hu/nju-thesis
%%
%% This file may be distributed and/or modified under the conditions of the
%% LaTeX Project Public License, either version 1.2 of this license or (at your
%% option) any later version. The latest version of this license is in:
%%
%% http://www.latex-project.org/lppl.txt
%%
%% and version 1.2 or later is part of all distributions of LaTeX version
%% 1999/12/01 or later.
%%
%%%%%%%%%%%%%%%%%%%%%%%%%%%%%%%%%%%%%%%%%%%%%%%%%%%%%%%%%%%%%%%%%%%%%%%%%%%%%%%


\begin{preface}

\end{preface}

% 生成论文目录
\tableofcontents
% 生成表格目录。如无需表格目录则可注释掉下述语句。
\listoftables
% 生成插图目录。如无需插图目录则可注释掉下述语句。
\listoffigures
%%%%%%%%%%%%%%%%%%%%%%%%%%%%%%%%%%%%%%%%%%%%%%%%%%%%%%%%%%%%%%%%%%%%%%%%%%%%%%%
% 开始正文部分
\mainmatter
% 下面是正文的各章节
%%%%%%%%%%%%%%%%%%%%%%%%%%%%%%%%%%%%%%%%%%%%%%%%%%%%%%%%%%%%%%%%%%%%%%%%%%%%%%%
%%  
%% 文档类 NJU-Thesis 用户手册
%%
%% 作者:胡海星,starfish (at) gmail (dot) com
%% 项目主页: https://github.com/Haixing-Hu/nju-thesis
%%
%% This file may be distributed and/or modified under the conditions of the
%% LaTeX Project Public License, either version 1.2 of this license or (at your
%% option) any later version. The latest version of this license is in:
%%
%% http://www.latex-project.org/lppl.txt
%%
%% and version 1.2 or later is part of all distributions of LaTeX version
%% 1999/12/01 or later.
%%
%%%%%%%%%%%%%%%%%%%%%%%%%%%%%%%%%%%%%%%%%%%%%%%%%%%%%%%%%%%%%%%%%%%%%%%%%%%%%%%

\chapter{绪论}\label{chapter:introduction}

\section{引言}

\section{学位论文排版的要求}

\section{现有的解决方案}

\section{现有方案的不足}

\section{本工作的意义和价值}

%%%%%%%%%%%%%%%%%%%%%%%%%%%%%%%%%%%%%%%%%%%%%%%%%%%%%%%%%%%%%%%%%%%%%%%%%%%%%%%
%%  
%% 文档类 NJU-Thesis 用户手册
%%
%% 作者:胡海星,starfish (at) gmail (dot) com
%% 项目主页: https://github.com/Haixing-Hu/nju-thesis
%%
%% This file may be distributed and/or modified under the conditions of the
%% LaTeX Project Public License, either version 1.2 of this license or (at your
%% option) any later version. The latest version of this license is in:
%%
%% http://www.latex-project.org/lppl.txt
%%
%% and version 1.2 or later is part of all distributions of LaTeX version
%% 1999/12/01 or later.
%%
%%%%%%%%%%%%%%%%%%%%%%%%%%%%%%%%%%%%%%%%%%%%%%%%%%%%%%%%%%%%%%%%%%%%%%%%%%%%%%%

\chapter{文档类的安装}\label{chapter:installtion}

\section{\TeX 系统的安装}

\section{\njuthesis 的安装}

\section{\njuthesis 的测试}

\section{\njuthesis 的升级}

%%%%%%%%%%%%%%%%%%%%%%%%%%%%%%%%%%%%%%%%%%%%%%%%%%%%%%%%%%%%%%%%%%%%%%%%%%%%%%%
%%  
%% 文档类 NJU-Thesis 用户手册
%%
%% 作者:胡海星,starfish (at) gmail (dot) com
%% 项目主页: https://github.com/Haixing-Hu/nju-thesis
%%
%% This file may be distributed and/or modified under the conditions of the
%% LaTeX Project Public License, either version 1.2 of this license or (at your
%% option) any later version. The latest version of this license is in:
%%
%% http://www.latex-project.org/lppl.txt
%%
%% and version 1.2 or later is part of all distributions of LaTeX version
%% 1999/12/01 or later.
%%
%%%%%%%%%%%%%%%%%%%%%%%%%%%%%%%%%%%%%%%%%%%%%%%%%%%%%%%%%%%%%%%%%%%%%%%%%%%%%%%

\chapter{学位论文的排版}

%%%%%%%%%%%%%%%%%%%%%%%%%%%%%%%%%%%%%%%%%%%%%%%%%%%%%%%%%%%%%%%%%%%%%%%%%%%%%%%
\section{\LaTeX 排版}


%%%%%%%%%%%%%%%%%%%%%%%%%%%%%%%%%%%%%%%%%%%%%%%%%%%%%%%%%%%%%%%%%%%%%%%%%%%%%%%
\section{学位论文的结构}


%%%%%%%%%%%%%%%%%%%%%%%%%%%%%%%%%%%%%%%%%%%%%%%%%%%%%%%%%%%%%%%%%%%%%%%%%%%%%%%


%%%%%%%%%%%%%%%%%%%%%%%%%%%%%%%%%%%%%%%%%%%%%%%%%%%%%%%%%%%%%%%%%%%%%%%%%%%%%%%
%%  
%% 文档类 NJU-Thesis 用户手册
%%
%% 作者:胡海星,starfish (at) gmail (dot) com
%% 项目主页: https://github.com/Haixing-Hu/nju-thesis
%%
%% This file may be distributed and/or modified under the conditions of the
%% LaTeX Project Public License, either version 1.2 of this license or (at your
%% option) any later version. The latest version of this license is in:
%%
%% http://www.latex-project.org/lppl.txt
%%
%% and version 1.2 or later is part of all distributions of LaTeX version
%% 1999/12/01 or later.
%%
%%%%%%%%%%%%%%%%%%%%%%%%%%%%%%%%%%%%%%%%%%%%%%%%%%%%%%%%%%%%%%%%%%%%%%%%%%%%%%%

\chapter{参考文献数据库}\label{chap:bib}

%%%%%%%%%%%%%%%%%%%%%%%%%%%%%%%%%%%%%%%%%%%%%%%%%%%%%%%%%%%%%%%%%%%%%%%%%%%%%%%
\section{简介}\label{sec:bib-intro}

%%%%%%%%%%%%%%%%%%%%%%%%%%%%%%%%%%%%%%%%%%%%%%%%%%%%%%%%%%%%%%%%%%%%%%%%%%%%%%%
\section{术语和定义}\label{sec:bib-glossary}

本节将描述在{\njuthesis}中用到的参考文献相关术语和定义\cite{gbt7714-2005}。

\subsection{文后参考文献(bibliographic references)}

为撰写或编辑论文和著作而引用的有关文献信息资源。

\subsection{主要贵任者(primary responsibility)}

对文献的知识内容或艺术内容负主要责任的个人或团体。主要责任者包括著者、编者、学位论文撰
写者、专利申请者或所有者、报告撰写者、标准提出者、析出文献的作者等。 

\subsection{专著(monographs)}

专著是指以单行本形式或多卷册形式,在限定的期限内出版的非连续性出版物。它包括以各种载
体形式出版的普通图书、古籍、学位论文、技术报告、会议文集、汇编、多卷书、丛书等。

在{\njuthesis}中,“专著”对应的{\BibTeX}文献项类型包括:
\begin{itemize}
\item 书籍(|book|),参见\ref{subsec:bibtype-book};
\item 会议录(|proceedings|或|conference|),参见\ref{subsec:bibtype-conference};
\item 汇编(|collection|),参见\ref{subsec:bibtype-collection};
\item 学位论文(|phdthesis|、|masterthesis|或|bachelorthesis|),
 参见\ref{subsec:bibtype-thesis};
\item 科技报告(|techreport|),参见\ref{subsec:bibtype-techreport};
\item 技术标准(|standard|),参见\ref{subsec:bibtype-standard};
\item 参考工具(|reference|),参见\ref{subsec:bibtype-reference};
\item 手册(|manaul|),参见\ref{subsec:bibtype-manual};
\item 未发表文献(|unpublised|或|manuscript|),参见\ref{subsec:bibtype-unpublished}。
\end{itemize}

以上各文献项都将按照《GB/T 7714-2005 文后参考文献着录规则》中“专著”所对应的
引用文献格式进行排版\cite{gbt7714-2005},但各项的必需字段和可选字段可能会有
所不同。

\subsection{连续出版物(serials)}

连续出版物是指一种载有卷期号或年月顺序号、计划无限期地连续出版发行的出版物。它包括以
各种载体形式出版的期刊、报纸、杂志等。

在{\njuthesis}中,“连续出版物”对应的{\BibTeX}文献项类型包括:
\begin{itemize}
\item 期刊(|periodical|),参见\ref{subsec:bibtype-periodical};
\item 报纸(|newspaper|),参见\ref{subsec:bibtype-newspaper};
\end{itemize}

以上各文献项都将按照《GB/T 7714-2005 文后参考文献着录规则》中“连续出版物”所对应的
引用文献格式进行排版\cite{gbt7714-2005},但各项的必需字段和可选字段可能会有
所不同。

\subsection{析出文献(contribution)}

析出文献是指从整本文献中析出的具有独立篇名的文献。 

在{\njuthesis}中,“析出文献”对应的{\BibTeX}文献项类型包括:
\begin{itemize}
\item 书籍中的析出文献(|inbook|),参见\ref{subsec:bibtype-inbook};
\item 会议录中的析出文献(|inproceedings|),参见\ref{subsec:bibtype-inproceedings};
\item 汇编中的析出文献(|incollection|),参见\ref{subsec:bibtype-incollection};
\item 期刊中的析出文献(|article|),参见\ref{subsec:bibtype-article};
\item 新闻报道(|news|),参见\ref{subsec:bibtype-news}。
\end{itemize}

以上各文献项都将按照《GB/T 7714-2005 文后参考文献着录规则》中“析出文献”所对应的
引用文献格式进行排版\cite{gbt7714-2005},但各项的必需字段和可选字段可能会有
所不同。

\subsection{电子文献(electronic documents)}

以数字方式将图、文、声、像等信息存储在磁、光、电介质上,通过计算机、网络或相关设备使用的记
录有知识内容或艺术内容的文献信息资源,包括电子书刊、数据库、电子公告等。

在{\njuthesis}中,“电子文献”对应的{\BibTeX}文献项类型包括:
\begin{itemize}
\item 网页(|online|),参见\ref{subsec:bibtype-online};
\item 计算机程序(|program|),参见\ref{subsec:bibtype-program};
\item 数据库(|database|),参见\ref{subsec:bibtype-database}。
\end{itemize}

以上各文献项都将按照《GB/T 7714-2005 文后参考文献着录规则》中“电子文档”所对应的
引用文献格式进行排版\cite{gbt7714-2005},但各项的必需字段和可选字段可能会有
所不同。

%%%%%%%%%%%%%%%%%%%%%%%%%%%%%%%%%%%%%%%%%%%%%%%%%%%%%%%%%%%%%%%%%%%%%%%%%%%%%%%
\section{格式}\label{sec:bib-format}

%%%%%%%%%%%%%%%%%%%%%%%%%%%%%%%%%%%%%%%%%%%%%%%%%%%%%%%%%%%%%%%%%%%%%%%%%%%%%%%
\section{文献类型}\label{sec:bib-type}

本节将描述{\njuthesis}文档类所支持的参考文献类型。

\begin{note}
因为要遵循\std{GB/T 7714-2005}规范\cite{gbt7714-2005},所以本节所描述的文献类型
可能和标准{\BibTeX}的文献类型有所差别。
\end{note}

\subsection{书籍}\label{subsec:bibtype-book}

书籍是专著的一种,包括普通图书、多卷书、丛书、教材等\cite{gbt3469-1983,gbt7714-2005}。
它所对应的{\BibTeX}文献项类型为|book|;对应的文献类型标志代码为|M|\cite{gbt3469-1983}。

\begin{note}
“书籍”和“汇编”的区别在于:“书籍”有自己的章节结构;而“汇编”没有;“汇编”中的每篇文献都有独立的标题;
而“书籍”没有,只有章节标题。
\end{note}

\subsubsection{必需字段}

\begin{itemize}
\item |author|:表示该书籍的作者,其格式参见\ref{subsec:bibfield-author};
\item |editor|:表示该书籍的编辑者,其格式参见\ref{subsec:bibfield-editor};
\item |title|:表示该书籍的标题,其格式参见\ref{subsec:bibfield-title};
\item |year|:表示该书籍的出版年,其格式参见\ref{subsec:bibfield-year}。
\end{itemize}

\begin{note}
|author|字段和|editor|字段应该至少有一个存在,但也可同时存在。若|author|字段存在,
{\BibTeX}将使用|author|字段的值作为该书籍的主要责任者;否则,若|editor|字段存在,
{\BibTeX}将使用|editor|字段的值作为该书籍的主要责任者;若|author|字段和|editor|字段
都不存在,{\BibTeX}将使用``Anon''或``佚名''作为该书籍的主要责任者;若|author|字段
和|editor|字段都存在,{\BibTeX}将使用|author|字段的值作为该书籍的主要责任者,而忽略
|editor|字段。
\end{note}

\subsubsection{可选字段}

\begin{itemize}
\item |translator|:表示该书籍的翻译者,其格式参见\ref{subsec:bibfield-translator};
\item |series|:表示该书籍所属的丛书的标题,其格式参见\ref{subsec:bibfield-series};
\item |edition|:表示该书籍的版本,其格式参见\ref{subsec:bibfield-edition};
\item |volume|:表示该书籍所属丛书或多卷书的卷号,其格式参见\ref{subsec:bibfield-volume};
\item |publisher|:表示该书籍的出版社,其格式参见\ref{subsec:bibfield-publisher};
\item |address|:表示该书籍的出版地,其格式参见\ref{subsec:bibfield-address};
\item |pages|:表示引文在该书籍中所处的页码或页码范围,其格式参见\ref{subsec:bibfield-pages};
\item |citedate|:表示该书籍的在线版本的引用日期,其格式参见\ref{subsec:bibfield-citedate};
\item |url|:表示该书籍的在线版本的引用URL,其格式参见\ref{subsec:bibfield-url};
\item |doi|:表示该书籍的电子版的DOI编号,其格式参见\ref{subsec:bibfield-doi};
\item |language|:表示该书籍的语言,其格式参见\ref{subsec:bibfield-language};
\item 其他字段将不起作用。
\end{itemize}

\begin{note}
书籍所属丛书的标题不应出现在|title|字段中,而应填写在|series|字段中。
书籍所属多卷书的卷号也不应出现在|title|字段中,而应填写在|volume|字段中。
若该书籍对应的文献项有|url|或|doi|字段,则其也必须有|citedate|字段。
若该书籍的语言为中文,则必须将|language|字段设置为|zh|;否则可忽略|language|字段。
\end{note}

\subsubsection{例子}

\begin{verbatim}
@book{lamport1994,
  author={Leslie Lamport},
  title={LaTeX A Document Preparation System: User's Guide 
         and Reference Manual},
  year={1994},
  edition={2},
  address={Reading, Massachusetts},
  publisher={Addison-Wesley},
}

@book{asperti1991,
  title={Categories, Types and Structures: an introduction 
         to Category Theory for the working computer scientist},
  author={Andrea Asperti and Giuseppe Longo},
  publisher={{MIT} Press},
  year={1991},
  citedate={2013-08-30},
  url={http://www.di.ens.fr/users/longo/download.html},
}

@book{takeuti1973,
  title={Axiomatic Set Theory},
  author={Gaisi Takeuti and Wilson M. Zaring},
  series={Graduate Texts in Mathematics},
  volume={8},
  editor={P. R. Halmos},
  address={Berlin},
  publisher={Springer-Verlag},
  year={1973},
}

@book{knuth1998,
 author={Knuth, Donald E.},
 title={Sorting and searching},
 series={The art of computer programming},
 volume={3},
 edition={2},
 year = {1998},
 isbn = {0-201-89685-0},
 publisher = {Addison Wesley},
 address = {Redwood},
} 

@book{yu2001,
  author={余敏 and 刘华},
  title={出版集团研究},
  address={北京},
  year={2001},
  pages={179--193},
  language={zh},
}

@book{anwen1988,
  author={昂温, G. and 昂温, P. S.},
  title={外国出版史},
  translator={陈生铮},
  edition={3},
  address={北京},
  publisher={中国书籍出版社},
  year={1988},
  citedate={2013-08-30},
  url={http://www.google.com/},
  language={zh},  
}

@book{guangxi1993,
  author={{广西壮族自治区林业厅}},
  title={广西自然保护区},
  address={北京},
  publisher={中国林业出版社},
  year={1993},
  language={zh},
}

@book{chuban2011,
  editor={{全国出版专业职业资格考试办公室}},
  title={出版专业实务(初级)},
  edition={2011年版},
  series={全国出版专业职业资格考试辅导教材},
  address={湖北},
  publisher={长江出版集团},
  year={2011},
  language={zh},
}
\end{verbatim}


\subsection{汇编}\label{subsec:bibtype-collection}

汇编是专著的一种。它是指汇总编辑而成的专著\cite{zdic2013huibian},例如关于某一主题的文章的合集,
或同一作者的文献集合之类。汇编包括论文集等\cite{gbt3469-1983}。它所对应的{\BibTeX}文献
项类型为|collection|;所对应的文献类型标志代码为|G|\cite{gbt3469-1983}。

\begin{note}
“论文集”应使用|collection|类型;而“会议录”和“会议论文集”应使用|proceedings|
或|conference|类型,参见\ref{subsec:bibtype-proceedings}。
\end{note}

\begin{note}
注意要区分“汇编”和“书籍”。一个简单的办法是看其结构。如果该文献是由章节构成,
则该文献属于“书籍”;即使每章作者不同,该文献也属于拥有多个作者的书籍。如果
该文献没有章节结构,而是由多篇独立的文献构成,每篇独立的文献都有自己的独立标题,
则该文献属于“汇编”;即使每篇独立文献的作者都相同,该文献也属于“汇编”。
\end{note}

\subsubsection{必需字段}

\begin{itemize}
\item |author|:表示该汇编的作者,其格式参见\ref{subsec:bibfield-author};
\item |editor|:表示该汇编的编辑者,其格式参见\ref{subsec:bibfield-editor};
\item |title|:表示该汇编的标题,其格式参见\ref{subsec:bibfield-title};
\item |year|:表示该汇编的出版年,其格式参见\ref{subsec:bibfield-year}。
\end{itemize}

\begin{note}
|author|字段和|editor|字段应该至少有一个存在,但也可同时存在。若|author|字段存在,
{\BibTeX}将使用|author|字段的值作为该汇编的主要责任者;否则,若|editor|字段存在,
{\BibTeX}将使用|editor|字段的值作为该汇编的主要责任者;若|author|字段和|editor|字段
都不存在,{\BibTeX}将使用``Anon''或``佚名''作为该汇编的主要责任者;若|author|字段
和|editor|字段都存在,{\BibTeX}将使用|author|字段的值作为该汇编的主要责任者,而忽略
|editor|字段。
\end{note}

\subsubsection{可选字段}

\begin{itemize}
\item |translator|:表示该汇编的翻译者,其格式参见\ref{subsec:bibfield-translator};
\item |series|:表示该汇编所属的丛书的标题,其格式参见\ref{subsec:bibfield-series};
\item |edition|:表示该汇编的版本,其格式参见\ref{subsec:bibfield-edition};
\item |volume|:表示该汇编所属丛书或多卷书的卷号,其格式参见\ref{subsec:bibfield-volume};
\item |publisher|:表示该汇编的出版社,其格式参见\ref{subsec:bibfield-publisher};
\item |address|:表示该汇编的出版地,其格式参见\ref{subsec:bibfield-address};
\item |pages|:表示引文在该汇编中所处的页码或页码范围,其格式参见\ref{subsec:bibfield-pages};
\item |citedate|:表示该汇编的在线版本的引用日期,其格式参见\ref{subsec:bibfield-citedate};
\item |url|:表示该汇编的在线版本的引用URL,其格式参见\ref{subsec:bibfield-url};
\item |doi|:表示该汇编的电子版的DOI编号,其格式参见\ref{subsec:bibfield-doi};
\item |language|:表示该汇编的语言,其格式参见\ref{subsec:bibfield-language};
\item 其他字段将不起作用。
\end{itemize}

\begin{note}
汇编所属丛书的标题不应出现在|title|字段中,而应填写在|series|字段中。
汇编所属多卷书的卷号也不应出现在|title|字段中,而应填写在|volume|字段中。
若该汇编对应的文献项有|url|或|doi|字段,则其也必须有|citedate|字段。
若该汇编的语言为中文,则必须将|language|字段设置为|zh|;否则可忽略|language|字段。
\end{note}

\subsubsection{例子}

\begin{verbatim}
@collection{engesser2009,
  title={Quantum Logic},
  series={Handbook of Quantum Logic and Quantum Structures},
  editor={Kurt Engesser and Dov M Gabbay and Daniel Lehmann},
  publisher={Elsevier},  
  year={2009},  
}

@collection{maxis1982,
  title={马克思恩格斯全集},
  author={马克思 and 恩格斯},  
  volume={44},
  address={北京},
  publisher={人民出版社},
  year={1982},
  language={zh},
}
\end{verbatim}

\subsection{会议录}\label{subsec:bibtype-conference}

会议录是专著的一种。它是指在一定范围的学术会议和专业性会议后,将会上宣读、讨论和散发的论文或报告,
加以编辑出版的文献\cite{hudong2013huiyilu}。会议也包括座谈会、讨论会等\cite{gbt7714-2005}。
会议录所对应的{\BibTeX}文献项类型为|proceedings|或|conference|;所对应的文献类型标志代
码为|C|\cite{gbt7714-2005}。

\begin{note}
中文中有时也将“会议录”称为“会议论文集”。注意“会议论文集”和一般意义上的“论文集”是不同的:
“会议论文集”就是指“会议录”,而“论文集”则是指关于某一主题的论文汇编,或同一个作者的论文
汇编。所以“会议录”或“会议论文集”应该使用|proceedings|或|conference|类型,对应的
\std{GB/T 3469-1983}文献类型单字码为|C|;而“论文集”应该使用|collection|类型,对应
的\std{GB/T 3469-1983}文献类型单字码为|G|。关于|collection|类型,
参见\ref{subsec:bibtype-collection}。
\end{note}

\subsubsection{必需字段}

\begin{itemize}
\item |editor|:表示该会议录的编辑者,其格式参见\ref{subsec:bibfield-editor};
\item |title|:表示该会议录的标题,其格式参见\ref{subsec:bibfield-title};
\item |year|:表示该会议录的出版年,其格式参见\ref{subsec:bibfield-year}。
\end{itemize}

\begin{note}
会议录的主要责任者应为其编辑者,应填写在|editor|字段中,而|author|字段将被忽略。
\end{note}

\subsubsection{可选字段}

\begin{itemize}
\item |translator|:表示该会议录的翻译者,其格式参见\ref{subsec:bibfield-translator};
\item |series|:表示该会议录所属的丛书的标题,其格式参见\ref{subsec:bibfield-series};
\item |volume|:表示该会议录在其所属的丛书或多卷书中的卷号,其格式参见\ref{subsec:bibfield-volume};
\item |edition|:表示该会议录的版本,其格式参见\ref{subsec:bibfield-edition};
\item |address|:表示该会议录的出版地,其格式参见\ref{subsec:bibfield-address};
\item |publisher|:表示该会议录的出版社,其格式参见\ref{subsec:bibfield-publisher};
\item |pages|:表示引文在该会议录中所处的页码或页码范围,其格式参见\ref{subsec:bibfield-pages};
\item |citedate|:表示该会议录的在线版本的引用日期,其格式参见\ref{subsec:bibfield-citedate};
\item |url|:表示该会议录的在线版本的引用URL,其格式参见\ref{subsec:bibfield-url};
\item |doi|:表示该会议录的DOI编码,其格式参见\ref{subsec:bibfield-doi};
\item |language|:表示该会议录的语言,其格式参见\ref{subsec:bibfield-language};
\item 其他字段将不起作用。
\end{itemize}

\begin{note}
会议录所属丛书的标题不应出现在|title|字段中,而应填写在|series|字段中。
会议录所属多卷书的卷号也不应出现在|title|字段中,而应填写在|volume|字段中。
若该会议录对应的文献项有|url|或|doi|字段,则其也必须有|citedate|字段。
若该会议录的语言为中文,则必须将|language|字段设置为|zh|;否则可忽略|language|字段。
\end{note}

\subsubsection{例子}

\begin{verbatim}
@proceedings{yufin2000,
  editor={Yufin, S. A.},
  title={Geoecology and computers: Proceedings of the Third 
         International Conference on Advance of Computer Methods 
         in Geotechnical and Geoenvironmental Engineering, 
         Moscow, Russia, February 1--4, 2000},
  address={Rotterdam},
  publisher={A. A. Balkema},
  year={2000},
}

@proceedings{rosenthall1963,
  editor={Rosenthall, E. M.},
  title={Proceedings of the Fifth Canadian Mathematical Congress,
         University of Montreal, 1961},
  address = {Toronto},
  publisher = {University of Toronto Press},
  year = {1963},
}

@conference{ganzha2000,
  editor = {Ganzha, V. G. and Mayr, E. W. and Vorozhtsov, E. V.},
  title = {Computer algebra in scientific computing: Proceedings
       of the Third Workshop on Algebra in Scientific Computing, 
       Samarkand, October 5-9,2000},
  address = {Berlin},
  publisher = {Springer},
  year = {c2000},
}

@proceedings{zhlixue1990,
  editor={{中国力学学会}},
  title={第3届全国实验流体力学学术会议论文集},
  address={天津},
  year={1990},
  language={zh},
}
\end{verbatim}


\subsection{学位论文}\label{subsec:bibtype-thesis}

学位论文是专著的一种。它所对应的{\BibTeX}文献项类型为|phdthesis|、|masterthesis|或
|bachelorthesis|,分别表示博士学位论文,硕士学位论文和学士学位论文;它所对应的
文献类型标志代码为|D|\cite{gbt3469-1983}。

|bachelorthesis|不是{\BibTeX}的标准类型,而是{\njuthesis}的扩展。

\subsubsection{必需字段}

\begin{itemize}
\item |author|:表示该学位论文的作者,其格式参见\ref{subsec:bibfield-author};
\item |title|:表示该学位论文的标题,其格式参见\ref{subsec:bibfield-title};
\item |school|:表示该学位论文所申请的学位的颁发单位,其格式参见\ref{subsec:bibfield-school};
\item |year|:表示该学位论文的所申请的学位的颁发年,其格式参见\ref{subsec:bibfield-year}。
\end{itemize}

\subsubsection{可选字段}

\begin{itemize}
\item |translator|:表示该学位论文的翻译者,其格式参见\ref{subsec:bibfield-translator};
\item |address|:表示该学位论文所申请的学位的颁发地,其格式参见\ref{subsec:bibfield-address};
\item |citedate|:表示该学位论文的在线版本的引用日期,其格式参见\ref{subsec:bibfield-citedate};
\item |url|:表示该学位论文的在线版本的引用URL,其格式参见\ref{subsec:bibfield-url};
\item |doi|:表示该学位论文的电子版的DOI编号,其格式参见\ref{subsec:bibfield-doi};
\item |language|:表示学位论文的语言,其格式参见\ref{subsec:bibfield-language};
\item 其他字段将不起作用。
\end{itemize}

\begin{note}
学位论文所对应的文献项用“学位颁发单位”(|school|字段)替代专著的“出版社”(|publisher|字段);
用“学位颁发地”(|address|字段)替代专著的“出版地”(|address|字段);用“学位颁发年”(|year|字段)
替代专著的“出版年”(|year|字段)。
若该学位论文对应的文献项有|url|或|doi|字段,则其也必须有|citedate|字段。
若该学位论文的语言为中文,则必须将|language|字段设置为|zh|;否则可忽略|language|字段。
\end{note}

\subsubsection{例子}

\begin{verbatim}
@masterthesis{calms1965,
  author={Calms, R. B.},
  title={Infrared spectroscopic studies on solid oxygen},
  address={Berkeley},
  school={University of California},
  year={1965},
}

@phdthesis{anderson1993,
  author={Penny Anderson},
  title={Program Derivation by Proof Transformation},
  address={Pittsburgh},
  school={Carnegie Mellon University},
  year={1993},
  citedate={2007-11-02},
  url={http://citeseer.nj.nec.com/anderson93program.html},
}

@phdthesis{sun2000,
  author={孙玉文},
  title={汉语变调构词研究},
  address={北京},
  school={北京大学中文系},
  year={2000},
  language={zh},
}

@masterthesis{zhang1998,
  author={张志祥},
  title={间断动力系统的随机扰动及其在守恒律方程中的应用},
  school={南京大学数学系},
  year={1998},
  language={zh},
}
\end{verbatim}

\subsection{科技报告}\label{subsec:bibtype-techreport}

科技报告是专著的一种。它包括科研报告、技术报告、调查报告、考察报告等\cite{gbt3469-1983}。
科技报告所对应的{\BibTeX}文献项类型为|techreport|;所对应的文献类型标志代码为|R|\cite{gbt3469-1983}。

\subsubsection{必需字段}

\begin{itemize}
\item |author|:表示该科技报告的作者,其格式参见\ref{subsec:bibfield-author};
\item |title|:表示该科技报告的标题,其格式参见\ref{subsec:bibfield-title};
\item |year|:表示该科技报告的发布年,其格式参见\ref{subsec:bibfield-year}。
\end{itemize}

\subsubsection{可选字段}

\begin{itemize}
\item |edition|:表示该科技报告的版本,其格式参见\ref{subsec:bibfield-edition};
\item |translator|:表示该科技报告的翻译者,其格式参见\ref{subsec:bibfield-translator};
\item |institution|:表示该科技报告的发布单位单位,其格式参见\ref{subsec:bibfield-institution};
\item |address|:表示该科技报告的发布单位的地址,其格式参见\ref{subsec:bibfield-address};
\item |citedate|:表示该科技报告的在线版本的引用日期,其格式参见\ref{subsec:bibfield-citedate};
\item |url|:表示该科技报告在线版本的引用URL,其格式参见\ref{subsec:bibfield-url};
\item |doi|:表示该科技报告的电子版的DOI编号,其格式参见\ref{subsec:bibfield-doi};
\item |language|:表示该科技报告的语言,其格式参见\ref{subsec:bibfield-language};
\item 其他字段将不起作用。
\end{itemize}

\begin{note}
科技报告所对应的文献项用“报告发布单位”(|institution|字段)替代专著的“出版社”(|publisher|字段);
用“报告发布单位的地址”(|address|字段)替代专著的“出版地”(|address|字段);用“报告发布年”(|year|字段)
替代专著的“出版年”(|year|字段)。
若该科技报告对应的文献项有|url|或|doi|字段,则其也必须有|citedate|字段。
若该科技报告的语言为中文,则必须将|language|字段设置为|zh|;否则可忽略|language|字段。
\end{note}

\subsubsection{例子}

\begin{verbatim}
@techreport{patashnik1988,
  author={Oren Patashnik},
  title={Designing \BibTeX{} Styles},
  year={1988},
}

@techreport{federal1990,
  author={{U.S. Department of Transportation 
           Federal Highway Administration}},
  title={Guidelines for handling excavated acid-producing 
        materials, PB 91-194001},
  address={Springfield},
  institution={U.S. Department of Commerce National 
      Information Service},
  year={1990},
}

@techreport{who1970,
  author={{World Health Organization}},
  title={Factors regulating the immune response: report 
     of WHO Scientific Group},
  address={Geneva},
  institution={WHO},
  year={1970},
}

@techreport{zhang2013,
  author={张三 and 李四 and 王五 and 赵六},
  title={使用非线性波导阵列实现量子随机游走中的理论和实践},
  institution={南京大学物理系},
  year={2013},
  language={zh},
}
\end{verbatim}


\subsection{技术标准}\label{subsec:bibtype-standard}

技术标准是专著的一种。它是指由某个团体或机构所定义的技术方面的标准。它所对应的{\BibTeX}文献项类
型为|standard|;所对应的文献类型标志代码为|S|\cite{gbt3469-1983}。

|standard|不是{\BibTeX}的标准类型,而是{\njuthesis}的扩展。

\subsubsection{必需字段}

\begin{itemize}
\item |author|:表示该技术标准的制定者,通常是一个机构名称,其格式参见\ref{subsec:bibfield-author};
\item |title|:表示该技术标准的名称,其格式参见\ref{subsec:bibfield-title};
\item |year|:表示该技术标准的出版年,其格式参见\ref{subsec:bibfield-year};
\end{itemize}

\begin{note}
技术标准的名称通常是由标准号加标准全称构成,标准号和标准全称之间应该用冒号``:''隔开;
\end{note}

\subsubsection{可选字段}

\begin{itemize}
\item |translator|:表示该技术标准的翻译者,其格式参见\ref{subsec:translator};
\item |edition|:表示该技术标准的版本,其格式参见\ref{subsec:bibfield-edition};
\item |address|:表示该技术标准的出版地,其格式参见\ref{subsec:bibfield-address};
\item |publisher|:表示该技术标准的出版者,其格式参见\ref{subsec:bibfield-publisher};
\item |pages|表示引文所处的页码或页码范围,其格式参见\ref{subsec:bibfield-pages};
\item |citedate|:表示该技术标准的在线版本的引用日期,其格式参见\ref{subsec:bibfield-citedate};
\item |url|:表示该技术标准的在线版本的引用URL,其格式参见\ref{subsec:bibfield-url};
\item |doi|:表示技术标准的电子版的DOI编号,其格式参见\ref{subsec:bibfield-doi};
\item |language|:表示技术标准的语言,其格式参见\ref{subsec:bibfield-language};
\item 其他字段将不起作用。
\end{itemize}


\begin{note}
若该技术标准对应的文献项有|url|或|doi|字段,则其也必须有|citedate|字段。
若该技术标准的语言为中文,则必须将|language|字段设置为|zh|;否则可忽略|language|字段。
\end{note}

\subsubsection{例子}

\begin{verbatim}
@standard{rfc2046,
  author={{IETF}},
  title={RFC 2046: Multipurpose Internet Mail Extensions 
        (MIME) Part Two: Media types},
  year={2010},
  url={http://tools.ietf.org/html/rfc2046},
  citedate={2013-10-20},
}

@standard{gbt7714-2005,
  author={{中国国家标准化管理委员会}},
  title={GB/T 7714-2005: 文后参考文献著录规则},
  year={2005},
  language={zh},
}

@standard{gbt7156-2003,
  author={{中国国家标准化管理委员会}},
  title={GB/T 7156-2003: 文献保密等级代码与标识},
  address={北京},
  publisher={科技出版社},
  year={2003},
  language={zh},
}
\end{verbatim}


\subsection{参考工具}\label{subsec:bibtype-reference}

参考工具是专著的一种。它包括年鉴、手册、百科全书、字典等\cite{gbt3469-1983}。它对应的
{\BibTeX}文献项类型为|reference|;所对应的文献类型标志代码为|K|\cite{gbt3469-1983}。

|reference|类型不是{\BibTeX}的标准类型,而是{\njuthesis}的扩展。

\begin{note}
在线版的字典、词典、百科全书等,应使用|reference|类型而非|online|类型。关于|online|类型,
参见\ref{subsec:bibtype-online}。
标准{\BibTeX}对于技术手册用|manual|类型,但在{\njuthesis}中\emph{也可}使
用|reference|类型。
\end{note}

\subsubsection{必需字段}

\begin{itemize}
\item |title|:表示该参考工具的标题,其格式参见\ref{subsec:bibfield-title}。
\end{itemize}

\subsubsection{可选字段}

\begin{itemize}
\item |author|:表示该参考工具的作者,其格式参见\ref{subsec:bibfield-author};
\item |editor|:表示该参考工具的编辑者,其格式参见\ref{subsec:bibfield-editor};
\item |translator|:表示该参考工具的翻译者,其格式参见\ref{subsec:translator};
\item |edition|:表示该参考工具的版本,其格式参见\ref{subsec:bibfield-edition};
\item |pages|表示引文在该参考工具中所处的页码或页码范围,其格式参见\ref{subsec:bibfield-pages};
\item |address|:表示该参考工具的出版地,其格式参见\ref{subsec:bibfield-address};
\item |publisher|:表示该参考工具的出版者,其格式参见\ref{subsec:bibfield-publisher};
\item |year|:表示该参考工具的出版年,其格式参见\ref{subsec:bibfield-year};
\item |citedate|:表示该参考工具的在线版本的引用日期,其格式参见\ref{subsec:bibfield-citedate};
\item |url|:表示该参考工具的在线版本的引用URL,其格式参见\ref{subsec:bibfield-url};
\item |doi|:表示该参考工具的电子版的DOI编号,其格式参见\ref{subsec:bibfield-doi};
\item |language|:表示该参考工具的语言,其格式参见\ref{subsec:bibfield-language};
\item 其他字段将不起作用。
\end{itemize}

\begin{note}
若该参考工具对应的文献项有|url|或|doi|字段,则其也必须有|citedate|字段。
若该参考工具的语言为中文,则必须将|language|字段设置为|zh|;否则可忽略|language|字段。
\end{note}

\subsubsection{例子}

\begin{verbatim}

@reference{webster,
  author={{Merriam-Webster}},  
  title={The Merriam-Webster English Dictionary},
  edition={revised edition},
  publisher={Merriam Webster Mass Market},
  year={2004},
}

@reference{xinhuazidian,
  author={{中国社会科学院语言研究所}},
  title={新华字典},
  edition={11},
  publisher={商务印书馆},
  address={北京},
  year={2011},
  language={zh},
}

@reference{hanyucidian,
  author={{中国社会科学院语言研究所词典编辑室}},
  title={现代汉语词典},
  edition={6},
  publisher={商务印书馆},
  address={北京},
  year={2012},
  language={zh},
}

@reference{shufabaike,
  author={李楠},
  title={书法历史},
  volume={1},
  series={中国书法百科全书},
  publisher={北京燕山出版社},
  address={北京},
  year={2010},
  language={zh},
}
\end{verbatim}


\subsection{手册}\label{subsec:bibtype-manual}

手册是一种特殊的参考工具。它所对应的{\BibTeX}文献项类型为|manual|;所对应的
文献类型标志代码为|K|\cite{gbt3469-1983}。

\subsubsection{必需字段}

\begin{itemize}
\item |title|:表示该手册的标题,其格式参见\ref{subsec:bibfield-title}。
\end{itemize}

\subsubsection{可选字段}

\begin{itemize}
\item |author|:表示该手册的作者,其格式参见\ref{subsec:bibfield-author};
\item |editor|:表示该手册的编辑者,其格式参见\ref{subsec:bibfield-editor};
\item |translator|:表示该手册的翻译者,其格式参见\ref{subsec:translator};
\item |edition|:表示该手册的版本,其格式参见\ref{subsec:bibfield-edition};
\item |pages|表示引文在该手册中所处的页码或页码范围,其格式参见\ref{subsec:bibfield-pages};
\item |address|:表示该手册的出版地,其格式参见\ref{subsec:bibfield-address};
\item |publisher|:表示该手册的出版者,其格式参见\ref{subsec:bibfield-publisher};
\item |year|:表示该手册的出版年,其格式参见\ref{subsec:bibfield-year};
\item |citedate|:表示该手册的在线版本的引用日期,其格式参见\ref{subsec:bibfield-citedate};
\item |url|:表示该手册的在线版本的引用URL,其格式参见\ref{subsec:bibfield-url};
\item |doi|:表示该手册的电子版的DOI编号,其格式参见\ref{subsec:bibfield-doi};
\item |language|:表示该手册的语言,其格式参见\ref{subsec:bibfield-language};
\item 其他字段将不起作用。
\end{itemize}

\begin{note}
若该手册对应的文献项有|url|或|doi|字段,则其也必须有|citedate|字段。
若该手册的语言为中文,则必须将|language|字段设置为|zh|;否则可忽略|language|字段。
\end{note}

\subsubsection{例子}

\begin{verbatim}
@manual{ipad,
  author={{Apple Inc}},
  title={iPad User Manual},
  address={Cupertino},
  publisher={Apple Inc},
  year={2008},
}

@manual{keynotes09,
  author={{Apple Inc}},
  title={Keynote'09 使用手册},
  address={Cupertino},
  publisher={Apple Inc},
  year={2009},
  language={zh},
}
\end{verbatim}


\subsection{期刊}\label{subsec:bibtype-periodical}

期刊是连续出版物的一种。它是指一种载有卷期号或年月顺序号、计划无限期地连续出版发行的出版
物\cite{gbt7714-2005}。它所对应的{\BibTeX}文献项类型为|periodical|;对应的
文献类型标志代码为|J|\cite{gbt3469-1983}。

|periodical|不是{\BibTeX}的标准类型,而是{\njuthesis}的扩展。之所以用|periodical|这
个名称而不是|journal|,是因为|journal|已经是{\BibTeX}的标准字段名,为避免重名冲突,
所以只能叫|periodical|。

\begin{note}
如果引文是期刊的某一期或某几期,则应使用|periodical|类型;如果引文是期刊中的文章,请使
用|article|类型,具体参见\ref{subsec:bibtype-article}。
杂志属于期刊的一种,但报纸和期刊有区别,因为他们对应的\std{GB/T 3469-1983}文献类型单
字码不同。报纸应采用|newspaper|文献项类型,参见\ref{subsec:bibtype-newspaper}。
\end{note}

\subsubsection{必需字段}

\begin{itemize}
\item |title|:表示该期刊的标题,其格式参见\ref{subsec:bibfield-title};
\item |year|表示所引用的期刊的出版年,或所引用的一系列连续期刊的出版年范围,其格式
  参见\ref{subsec:bibfield-year}。
\end{itemize}

\subsubsection{可选字段}

\begin{itemize}
\item |editor|:表示该期刊的编辑者,其格式参见\ref{subsec:bibfield-editor};
\item |translator|:表示该期刊的翻译者,其格式参见\ref{subsec:translator};
\item |volume|: 表示所引用的期刊的卷号,或所引用的一系列连续期刊的卷号范围,其格式参
  见\ref{subsec:bibfield-volume};
\item |number|: 表示所引用的期刊的期号,或所引用的一系列连续期刊的期号范围,其格式参
  见\ref{subsec:bibfield-number};
\item |publisher|:表示该期刊的出版者,其格式参见\ref{subsec:bibfield-publisher};
\item |address|:表示该期刊的出版地,其格式参见\ref{subsec:bibfield-address};  
\item |citedate|:表示所引用的期刊的在线版本的引用日期,其格式参见\ref{subsec:bibfield-citedate};
\item |url|:表示该期刊的在线版本的引用URL,其格式参见\ref{subsec:bibfield-url};
\item |doi|:表示该期刊的电子版的DOI编号,其格式参见\ref{subsec:bibfield-doi};
\item |language|:表示该期刊的语言,其格式参见\ref{subsec:bibfield-language};
\item 其他字段将不起作用。
\end{itemize}

\begin{note}
若该期刊对应的文献项有|url|或|doi|字段,则其必须也有|citedate|字段。
若该期刊的语言为中文,则必须将|language|字段设置为|zh|;否则可忽略|language|字段。
\end{note}

\subsubsection{例子}

\begin{verbatim}
@periodical{science1983,
  editor={{American Association for the Advancement of Science}},
  title={Science},
  year={1883},
  volume={1},
  number={1},
  address={Washington},
  publisher={American Association for the Advancement of Science},
}

@periodical{dizhipinlun1936,
  editor={{中国地质学会}},
  title={地质论评},
  year={1936},
  volume={1},
  number={1},
  address={北京},
  publisher={地质出版社},
  language={zh},
}

@periodical{tushuguanxutongxun1,
  editor={{中国图书馆学会}},
  title={图书馆学通讯},
  year={1957-1990},
  number={1-4},
  address={北京},
  publisher={北京图书馆},
  language={zh},
}

@periodical{tushuguanxuetongxun17,
  editor={{中国图书馆学会}},
  title={图书馆学通讯},
  year={1957-1990},
  volume={17-57},
  address={北京},
  publisher={北京图书馆},
  language={zh},
}
\end{verbatim}


\subsection{报纸}\label{subsec:bibtype-newspaper}

报纸是连续出版物的一种,它与期刊类似。它所对应的{\BibTeX}文献项类型为|newspaper|;对应的
文献类型标志代码为|N|\cite{gbt3469-1983}。

|newspaper|不是{\BibTeX}的标准类型,而是{\njuthesis}的扩展。之所以将|newspaper|和
|periodical|区分开,是因为期刊(|periodical|)所对应的文献类型单字码为|J|,而报纸
(|newspaper|)所对应的文献类型单字码为|N|\cite{gbt3469-1983}。

\begin{note}
如果引文是报纸的某一期或某几期,则应使用|newspaper|类型;如果引文是某一期报纸中的文章,
请使用|news|类型,具体参见\ref{subsec:bibtype-news}。
\end{note}

\subsubsection{必需字段}

\begin{itemize}
\item |title|:表示该报纸的标题,其格式参见\ref{subsec:bibfield-title};
\item |year|表示所引用的报纸的出版年,或所引用的一系列连续报纸的出版年范围,其格式
  参见\ref{subsec:bibfield-year}。
\end{itemize}

\subsubsection{可选字段}

\begin{itemize}
\item |editor|:表示该期刊的编辑者,其格式参见\ref{subsec:bibfield-editor};
\item |volume|: 表示所引用的报纸的卷号,或所引用的一系列连续报纸的卷号范围,其格式参
  见\ref{subsec:bibfield-volume};
\item |number|: 表示所引用的报纸的期号,或所引用的一系列连续报纸的期号范围,其格式参
  见\ref{subsec:bibfield-number};
\item |publisher|:表示该报纸的出版社,其格式参见\ref{subsec:bibfield-publisher};
\item |address|:表示该报纸的出版地,其格式参见\ref{subsec:bibfield-address};
\item |citedate|:表示所引用的报纸的在线版本的引用日期,其格式参见\ref{subsec:bibfield-citedate};
\item |url|:表示该报纸的在线版本的引用URL,其格式参见\ref{subsec:bibfield-url};
\item |doi|:表示该报纸的电子版的DOI编号,其格式参见\ref{subsec:bibfield-doi};
\item |language|:表示该报纸的语言,其格式参见\ref{subsec:bibfield-language};
\item 其他字段将不起作用。
\end{itemize}

\begin{note}
若该报纸对应的文献项有|url|或|doi|字段,则其也必须有|citedate|字段。
若该报纸的语言为中文,则必须将|language|字段设置为|zh|;否则可忽略|language|字段。
\end{note}

\subsubsection{例子}

\begin{verbatim}
@newspaper{financialtimes,
  title={The Financial Times},
  year={1888-1913},
  volume={1-512},
  number={1-1210}
  address={London},
  publisher={Pearson PLC},  
}

@newspaper{renminribao,
  editor={{人民日报编辑部}},
  title={人民日报},
  year={2011},
  volume={22892},
  address={北京},
  publisher={人民日报出版社},
  language={zh},
}
\end{verbatim}


\subsection{专利文献}\label{subsec:bibtype-patent}

专利文献对应的{\BibTeX}文献项类型为|patent|。它对应的文献类型标志代码为|P|\cite{gbt3469-1983}。

|patent|不是{\BibTeX}的标准类型,而是{\njuthesis}的扩展。

\subsubsection{必需字段}

\begin{itemize}
\item |author|:表示该专利的申请者或所有者,其格式参见\ref{subsec:bibfield-author};
\item |title|:表示该专利的名称,其格式参见\ref{subsec:bibfield-title};
\item |country|:表示该专利的国别,其格式参见\ref{subsec:bibfield-country};
\item |patentid|:表示该专利的专利号,其格式参加\ref{subsec:bibfield-patentid};
\item |date|:表示该专利的公告日期或公开日期,其格式参见\ref{subsec:bibfield-date}。
\end{itemize}

\subsubsection{可选字段}

\begin{itemize}
\item |citedate|:表示该专利文献的在线版本的引用日期,其格式参见\ref{subsec:bibfield-citedate};
\item |url|:表示该专利文献在线版本的引用URL,其格式参见\ref{subsec:bibfield-url};
\item |doi|:表示该专利文献的电子版的DOI编号,其格式参见\ref{subsec:bibfield-doi};
\item |language|:表示该专利文献的语言,其格式参见\ref{subsec:bibfield-language};
\item 其他字段将不起作用。
\end{itemize}

\begin{note}
若该专利文献对应的文献项有|url|或|doi|字段,则其也必须有|citedate|字段。
若该专利文献的语言为中文,则必须将|language|字段设置为|zh|;否则可忽略|language|字段。
\end{note}

\subsubsection{例子}

\begin{verbatim}
@patent{p6915001,
  author={Tachibana, R. and Shimizu, S. 
          and Kobayshi, S. and Other Authors},
  title={Electronic watermarking method and system},
  country={US},
  patentid={6,915,001},
  date={2002-04-25},
}

@patent{p8284102,
  author={Koseki, A. and Momose, H. 
          and Kawahito, M. and Other Authors},
  title={Compiler},
  country={US},
  patentid={8,284,102},
  date={2002-05-25},
}

@patent{p88105607.3,
  author={姜锡洲},
  title={一种温热外敷药制备方案},
  country={中国},
  patentid={88105607.3},
  date={1989-07-26},
  langauge={zh},
}

@patent{p01128777.2,
  author={{西安电子科技大学}},
  title={光折变自适应光外差探测方法},
  country={中国},
  patentid={01128777.2},
  date={2002-03-06},
  langauge={zh},
}

@patent{p92214985.2,
  author={刘加林},
  title={多功能一次性压舌板},
  country={中国},
  patentid={92214985.2},
  date={1993-04-14},
  language={zh},
}

@patent{p01129210.5,
  author={{河北绿洲生态环境科技有限公司}},
  title={一种荒漠化地区生态植被综合培育种植方法},
  country={中国},
  patentid={01129210.5},
  date={2001-10-24},
  language={zh},
}
\end{verbatim}

\subsection{网页}\label{subsec:bibtype-online}

网页是电子文献的一种。其所对应的{\BibTeX}文献项类型为|online|;所对应的文献类型编码
为|EB|\cite{gbt7714-2005}。

|online|不是{\BibTeX}的标准类型,而是{\njuthesis}的扩展。

\subsubsection{必需字段}

\begin{itemize}
\item |title|:表示该网页的标题,其格式参见\ref{subsec:bibfield-title};
\item |citedate|:表示引用该网页的日期,其格式参见\ref{subsec:bibfield-citedate};
\item |url|:表示该网页的URL,其格式参见\ref{subsec:bibfield-url}。
\end{itemize}

\subsubsection{可选字段}

\begin{itemize}
\item |author|:表示该网页的作者,可以是个人或单位,
    其格式参见\ref{subsec:bibfield-author};
\item |address|:表示该网页的发布者(通常是某个机构)的地址,
    其格式参见\ref{subsec:bibfield-address};
\item |publisher|:表示网页的发布者,通常是某个机构,
    其格式参见\ref{subsec:bibfield-publisher};
\item |year|:表示网页的发布年(首发年),
    其格式参见\ref{subsec:bibfield-year};
\item |modifydate|:表示网页的最后一次修改日期,
    其格式参见\ref{subsec:bibfield-modifydate};
\item |language|:表示网页的语言,
    其格式参见\ref{subsec:bibfield-language};
\item 其他字段将不起作用。
\end{itemize}

\begin{note}
网页所对应的文献类型编码为|EB|,表示“电子公告”,规定在\cite{gbt7714-2005}的附录B.1中。
网页没有|doi|字段,但必须有|url|和|citedate|字段。
若该网页的语言为中文,则必须将|language|字段设置为|zh|;否则可忽略|language|字段。
\end{note}

\subsubsection{例子}

\begin{verbatim}
@online{pacs1989,
  title={PACS-L: the public-access computer systems forum},
  address={Houston, Tex},
  publisher={University of Houston Libraries},
  year={1989},
  url={http://info.lib.uh.edu/pacsl.html},
  citedate={1995-05-17},
}

@online{oclc2000,
  author={{Online Computer Library Center, Inc.}},
  title={History of OCLC},
  url={http://www.oclc.org/about/history/default.htm},
  citedate={2000-01-08},
}

@online{chuban2001,
  author={萧钰},
  title={出版业信息化迈入快车道},
  modifydate={2001-12-19},
  url={http://www.creader.com/news/20011219/200112190019.html},
  citedate={2002-04-15},
  langauge={zh},
}

@online{h7n9,
  author={张乐},
  title={我科学家成功研发人感染H7N9禽流感病毒疫苗株},
  address={北京},
  publisher={人民网}, 
  year={2013},
  modifydate={2013-10-27},
  url={http://society.people.com.cn/n/2013/1027/c1008-23337665.html},
  citedate={2013-10-27},
  langauge={zh},
}
\end{verbatim}


\subsection{计算机程序}\label{subsec:bibtype-program}

计算机程序所对应的{\BibTeX}文献项类型为|program|;所对应的文献类型编码为|CP|\cite{gbt7714-2005}。

|program|不是{\BibTeX}的标准类型,而是{\njuthesis}的扩展。

\subsubsection{必需字段}

\begin{itemize}
\item |title|:表示该程序的名称,其格式参见\ref{subsec:bibfield-title};
\item |year|:表示该程序的发布年,其格式参见\ref{subsec:bibfield-year}。
\end{itemize}

\subsubsection{可选字段}

\begin{itemize}
\item |author|:表示该程序的作者,可以是个人或单位,其格式参见\ref{subsec:bibfield-author};
\item |address|:表示该程序的发布者(通常是某个机构)的地址,其格式参见
  \ref{subsec:bibfield-address};
\item |publisher|:表示该程序的发布者,通常是某个机构,其格式参见
  \ref{subsec:bibfield-publisher};
\item |citedate|:表示引用该程序的在线版的日期,其格式参见\ref{subsec:bibfield-citedate};
\item |url|:表示该程序的在线版的URL,其格式参见\ref{subsec:bibfield-url};
\item |doi|:表示该程序的电子版的DOI编号,其格式参见\ref{subsec:bibfield-doi};
\item |media|:表示该程序载体的介质,其格式参见\ref{subsec:bibfield-media};
\item |language|:表示该程序的名称、作者、发布者等属性所采用的语言;
其格式参见\ref{subsec:bibfield-language};
\item 其他字段将不起作用。
\end{itemize}

\begin{note}
若该程序对应的文献项有|url|或|doi|字段,则其也必须有|citedate|字段。
若|url|或|doi|字段存在,则|media|字段将被忽略;否则,|media|字段表示该程序载体的介质,其具体
编码参见\ref{subsec:bibfield-media}。
|language|字段不是指编写该程序的程序设计语言,而是指描述该程序的属性(名称、作者、发布者等)
所使用的语言;若该语言为中文,则必须将|language|字段设置为|zh|;否则可忽略|language|字段。
\end{note}

\subsubsection{例子}

\begin{verbatim}
@program{scheduler1983,
  title={Project scheduler},
  address={Sunnyvale, Calif.},
  publisher={Scitor Corporation},
  year={c1983},
  media={DK},
}

@program{njuthesis,
  author={胡海星},
  title={南京大学学位论文模板},
  year={2013},
  citedate={2013-08-31},
  url={https://github.com/Haixing-Hu/nju-thesis},
  language={zh},
}
\end{verbatim}


\subsection{数据库}\label{subsec:bibtype-database}

数据库所对应的{\BibTeX}文献项类型为|database|;所对应的文献类型编码为|DB|\cite{gbt7714-2005}。

|database|不是{\BibTeX}的标准类型,而是{\njuthesis}的扩展。

\subsubsection{必需字段}

\begin{itemize}
\item |title|:表示该数据库的名称,其格式参见\ref{subsec:bibfield-title};
\item |year|:表示该数据库的发布年,其格式参见\ref{subsec:bibfield-year}。
\end{itemize}

\subsubsection{可选字段}

\begin{itemize}
\item |author|:表示该数据库的作者,可以是个人或单位,其格式参见\ref{subsec:bibfield-author};
\item |address|:表示该数据库的发布者(通常是某个机构)的地址,其格式参见
  \ref{subsec:bibfield-address};
\item |publisher|:表示该数据库的发布者,通常是某个机构,其格式参见
  \ref{subsec:bibfield-publisher};
\item |citedate|:表示引用该数据库的在线版的日期,其格式参见\ref{subsec:bibfield-citedate};
\item |url|:表示该数据库的在线版的URL,其格式参见\ref{subsec:bibfield-url};
\item |doi|:表示该数据库的电子版的DOI编号,其格式参见\ref{subsec:bibfield-doi};
\item |media|:表示该数据库载体的介质,其格式参见\ref{subsec:bibfield-media};
\item |language|:表示该数据库的名称、作者、发布者等属性所采用的语言;
其格式参见\ref{subsec:bibfield-language};
\item 其他字段将不起作用。
\end{itemize}

\begin{note}
若该数据库对应的文献项有|url|或|doi|字段,则其也必须有|citedate|字段。
若|url|或|doi|字段存在,则|media|字段将被忽略;否则,|media|字段表示该数据库载体的介质,其具体
编码参见\ref{subsec:bibfield-media}。
|language|字段是指描述该数据库的属性(名称、作者、发布者等)所使用的语言;若该语言为中文,
则必须将|language|字段设置为|zh|;否则可忽略|language|字段。
\end{note}

\subsubsection{例子}

\begin{verbatim}
@database{planetosm,
  title={Planet OSM},
  publisher={OpenStreetMap.org},
  year={2013},
  url={http://planet.openstreetmap.org/},
  citedate={2013-10-27},  
}

@database{gaoxiao2011,
  author={{北京市教育委员会}},
  title={北京市高等院校信息},
  publisher={北京市政务数据资源网},
  address={北京},
  year={2011},
  modifydate={2011-07-06},
  url={http://www.bjdata.gov.cn/tabid/93/Default.aspx?did=228},
  citedate={2013-10-27},
  language={zh},
}
\end{verbatim}

\subsection{未发表文献}\label{subsec:bibtype-unpublished}

未发表文献所对应的{\BibTeX}文献项类型为|unpublished|或|manuscript|;它所对应的文献类型
编码为|H|\cite{gbt3469-1983}。

|manuscript|不是{\BibTeX}的标准类型,而是{\njuthesis}的扩展。

\subsubsection{必需字段}

\begin{itemize}
\item |author|:表示该文献的作者,其格式参见\ref{subsec:bibfield-author};
\item |title|:表示该文献的标题,其格式参见\ref{subsec:bibfield-title};
\item |year|:表示该文献的写作年,其格式参见\ref{subsec:bibfield-year}。
\end{itemize}

\subsubsection{可选字段}

\begin{itemize}
\item |pages|表示引文在该文献中所处的页码或页码范围,其格式参见\ref{subsec:bibfield-pages};  
\item |citedate|:表示引用该文献的在线版的日期,其格式参见\ref{subsec:bibfield-citedate};
\item |url|:表示该文献的在线版的URL,其格式参见\ref{subsec:bibfield-url};
\item |doi|:表示该文献的电子版的DOI编号,其格式参见\ref{subsec:bibfield-doi};
\item |language|:表示该文献的语言,其格式参见\ref{subsec:bibfield-language};
\item 其他字段将不起作用。
\end{itemize}

\begin{note}
若该文献对应的文献项有|url|或|doi|字段,则其也必须有|citedate|字段。
若该文献的语言为中文,则必须将|language|字段设置为|zh|;否则可忽略|language|字段。
\end{note}

\subsubsection{例子}

\begin{verbatim}
@unpublished{bove2002,
  author = {Ana Bove and Venanzio Capretta},
  title = {Modelling General Recursion in Type Theory},
  year = {2002},
  citedate={2007-10-11},
  url = {http://citeseer.nj.nec.com/bove02modelling.html},
}

@manuscript{quantum,
  author={胡海星},
  title={量子随机行走的理论和实现},
  year={2013},
}
\end{verbatim}


\subsection{书籍中的析出文献}\label{subsec:bibtype-inbook}

书籍中的析出文献是指书籍中某一具有独立标题和作者的章节。它所对应的{\BibTeX}文献
项类型为|inbook|;所对应的文献类型编码为|M|\cite{gbt3469-1983}。

\begin{note}
如果是汇编中的析出文献,请使用|incollection|类型,具体参见\ref{subsec:bibtype-incollection}。
这和标准{\BibTeX}中对|inbook|与|incollection|类型的定义有所不同。
\end{note}

\subsubsection{必需字段}

\begin{itemize}
\item |author|:表示该析出文献的作者,其格式参见\ref{subsec:bibfield-author};
\item |title|:表示该析出文献的标题,其格式参见\ref{subsec:bibfield-title};
\item |booktitle|:表示该析出文献所属书籍的标题,其格式参见\ref{subsec:bibfield-title};
\item |editor|:表示该析出文献所属书籍的编辑者或作者,其格式参见\ref{subsec:bibfield-editor};
\item |year|:表示该析出文献所属书籍的出版年,其格式参见\ref{subsec:bibfield-year};
\item |pages|:表示该析出文献在其所属书籍中所处的页码或页码范围,其格式参见\ref{subsec:bibfield-pages}。
\end{itemize}

\subsubsection{可选字段}

\begin{itemize}
\item |translator|:表示该析出文献或其所属书籍的翻译者,其格式参见\ref{subsec:translator};
\item |edition|:表示该析出文献所属书籍的版本,其格式参见\ref{subsec:bibfield-edition};
\item |address|:表示该析出文献所属书籍的出版地,其格式参见\ref{subsec:bibfield-address};
\item |publisher|:表示该析出文献所属书籍的出版者,其格式参见\ref{subsec:bibfield-publisher};
\item |citedate|:表示该析出文献的在线版本的引用日期,其格式参见\ref{subsec:bibfield-citedate};
\item |url|:表示该析出文献的在线版本的引用URL,其格式参见\ref{subsec:bibfield-url};
\item |doi|:表示该析出文献的电子版的DOI编号,其格式参见\ref{subsec:bibfield-doi};
\item |language|:表示该析出文献的语言,其格式参见\ref{subsec:bibfield-language};
\item 其他字段将不起作用。
\end{itemize}

\begin{note}
该书籍所属的丛书的标题不应出现在|booktitle|字段中,而应填写在|series|字段中。
该书籍所属的多卷书的卷号也不应出现在|booktitle|字段中,而应填写在|volume|字段中。
若该析出文献对应的文献项有|url|或|doi|字段,则其也必须有|citedate|字段。
若该析出文献的语言为中文,则必须将|language|字段设置为|zh|;否则可忽略|language|字段。
\end{note}

\subsubsection{例子}

\begin{verbatim}
@inbook{changjiang1999,
  author={程根伟},
  title={1998年长江洪水的成因与减灾对策},
  editor={许厚泽 and 赵其国},
  booktitle={长江流域洪涝灾害与科技对策},
  address={北京},
  publisher={科学出版社},
  year={1999},
  pages={32-36},
  language={zh},
}

@inbook{zhendan1980,
  author={陈晋镶 and 张惠民 and 朱士兴 and 其他作者},
  title={蓟县震旦亚界研究},
  editor={中国地质科学院天津地质矿产研究所},
  booktitle={中国震旦亚界},
  address={天津},
  publisher={天津科学技术出版社},
  year={1980},
  pages={56-114},
  language={zh},
}

@inbook{weinstein1974,
  author={Weinstein, L. and Swertz, M. N.},
  title={Pathogenic properties of invading microorganism},
  editor={Sodeman, Jr., W. A. and Sodeman, W. A.},
  booktitle={Pathologic physiology: mechanisms of disease},
  address={Philadelphia},
  publisher={Saunders},
  year={1974},
  pages={745-772},
}

@inbook{buseck1980,
  author={Buseck, P. R. and Nord, Jr., G. L. and Veblen D. R.},
  title={Subsolidus phenomena in pyroxenes},
  editor={Prewitt, C. T.},
  booktitle={Pyroxense},
  address={Washington, D. C.},
  publisher={Mineralogical Society of America},
  year={c1980},
  pages={117-211},
}

@inbook{martin1966,
  author={Martin, G.},
  title={Control of electronic resources in Australia},
  editor={Pattle, L. W. and Cox, B. J.},
  booktitle={Electronic resources: selection and 
   bibliographic control},
  address={New York},
  publisher={The Haworth Press},
  year={1966},
  pages={85-96},
}

@inbook{winskel1995,
  author={G. Winskel and M. Nielsen},
  title={Models for Concurrency},
  booktitle={Handbook of Logic in Computer Science},
  publisher={Oxford University Press},
  year={1995},
  editor={S. Abramsky and D. Gabbay and T. S. E. Maibaum},
  citedate={2013-10-20},
  url={http://citeseer.ist.psu.edu/winskel94models.html},
}

@inbook{abramsky1994,
  author={Samson Abramsky and Achim Jung},
  title={Domain Theory},
  booktitle={Semantic Structures},
  series={Handbook of Logic in Computer Science},
  volume={3},
  publisher={Oxford University Press},
  year={1994},
  editor={Samson Abramsky and Dov M. Gabbay 
    and T.S.E. Maibaum},
  pages={1-168},
}

@inbook{barendregt1992,
  author={Henk Barendregt},
  title={Lambda Calculi with Types},
  booktitle={Background: Computational Structures},
  series={Handbook of Logic in Computer Science},
  volume={2},
  publisher={Oxford University Press},
  year={1992},
  editor={Samson Abramsky and Dov M. Gabbay and T.S.E. Maibaum},
  pages={117-309},
}

@inbook{berger1995a,
  author={Ulrich Berger and Helmut Schwichtenberg},
  title={Program Development by Proof Transformation},
  booktitle={Proof and Computation},
  publisher={Springer},
  year={1995},
  editor={Helmut Schwichtenberg},
  pages={1-45},
  address={Berlin},
}

@inbook{buss1998introduction,
  author={Samuel R. Buss},
  title={An Introduction to Proof Theory},
  booktitle={Handbook Proof Theory},
  address={Amsterdam},
  publisher={Elsevier},
  year={1998},
  volume={I},
}

@inbook{buss1998first,
  author={Samuel R. Buss},
  title={First-Order Theory of Arithmetic},
  booktitle={Handbook Proof Theory},
  address={Amsterdam},
  publisher={Elsevier},
  year={1998},
  volume={II},
}
\end{verbatim}

\subsection{汇编中的析出文献}\label{subsec:bibtype-incollection}

汇编中的析出文献是指收录在汇编中的文献。它所对应的{\BibTeX}文献项类型为|incollection|;
所对应的文献类型编码为|C|\cite{gbt3469-1983}。

\begin{note}
如果是专著中的析出文献,请使用|inbook|类型,具体参见\ref{subsec:bibtype-inbook}。
这和标准{\BibTeX}中对|inbook|与|incollection|类型的定义有所不同。
\end{note}

\subsubsection{必需字段}

\begin{itemize}
\item |author|:表示该析出文献的作者,其格式参见\ref{subsec:bibfield-author};
\item |title|:表示该析出文献的标题,其格式参见\ref{subsec:bibfield-title};
\item |booktitle|:表示该析出文献所属汇编的标题,其格式参见\ref{subsec:bibfield-title};
\item |editor|:表示该析出文献所属汇编的编辑者或作者,其格式参见\ref{subsec:bibfield-editor};
\item |year|:表示该析出文献所属汇编的出版年,其格式参见\ref{subsec:bibfield-year};
\item |pages|:表示该析出文献在其所属汇编中所处的页码或页码范围,其格式参见\ref{subsec:bibfield-pages}。
\end{itemize}

\subsubsection{可选字段}

\begin{itemize}
\item |translator|:表示该析出文献或其所属汇编的翻译者,其格式参见\ref{subsec:translator};
\item |edition|:表示该析出文献所属汇编的版本,其格式参见\ref{subsec:bibfield-edition};
\item |address|:表示该析出文献所属汇编的出版地,其格式参见\ref{subsec:bibfield-address};
\item |publisher|:表示该析出文献所属汇编的出版者,其格式参见\ref{subsec:bibfield-publisher};
\item |citedate|:表示该析出文献的在线版本的引用日期,其格式参见\ref{subsec:bibfield-citedate};
\item |url|:表示该析出文献的在线版本的引用URL,其格式参见\ref{subsec:bibfield-url};
\item |doi|:表示该析出文献的电子版的DOI编号,其格式参见\ref{subsec:bibfield-doi};
\item |language|:表示该析出文献的语言,其格式参见\ref{subsec:bibfield-language};
\item 其他字段将不起作用。
\end{itemize}

\begin{note}
该汇编所属的丛书的标题不应出现在|booktitle|字段中,而应填写在|series|字段中。
该汇编所属的多卷书的卷号也不应出现在|booktitle|字段中,而应填写在|volume|字段中。
若该析出文献对应的文献项有|url|或|doi|字段,则其也必须有|citedate|字段。
若该析出文献的语言为中文,则必须将|language|字段设置为|zh|;否则可忽略|language|字段。
\end{note}

\subsubsection{例子}

\begin{verbatim}
@incollection{abramsky1990,
  author={Samson Abramsky},
  title={The Lazy Lambda Calculus},
  booktitle={Research Topics in Functional Programming},
  publisher={Addison-Welsey},
  year={1990},
  editor={D. A. Turner},
  pages={65-116},
  address={Reading, MA},
  citedate={2010-01-12},
  url={http://citeseer.nj.nec.com/abramsky90lazy.html},
}

@incollection{gongzi,
  author={马克思},
  title={关于《工资、价格和利润》的报告札记},
  editor={马克思 and 恩格斯},
  booktitle={马克思恩格斯全集},
  volume={44},
  address={北京},
  publisher={人民出版社},
  year={1982},
  pages={505},
  language={zh},
}
\end{verbatim}

\subsection{会议录中的析出文献}\label{subsec:bibtype-inproceedings}

会议录中的析出文献是指收录在会议录中的论文或报告。它所对应的{\BibTeX}文献项类型为
|inproceedings|;所对应的文献类型编码为|C|\cite{gbt3469-1983}。

\subsubsection{必需字段}

\begin{itemize}
\item |author|:表示该析出文献的作者,其格式参见\ref{subsec:bibfield-author};
\item |title|:表示该析出文献的标题,其格式参见\ref{subsec:bibfield-title};
\item |booktitle|:表示该析出文献所属会议录的标题,其格式参见\ref{subsec:bibfield-title};
\item |editor|:表示该析出文献所属会议录的编辑者或作者,其格式参见\ref{subsec:bibfield-editor};
\item |year|:表示该析出文献所属会议录的出版年,其格式参见\ref{subsec:bibfield-year};
\item |pages|:表示该析出文献在其所属会议录中所处的页码或页码范围,其格式参见\ref{subsec:bibfield-pages}。
\end{itemize}

\subsubsection{可选字段}

\begin{itemize}
\item |translator|:表示该析出文献或其所属会议录的翻译者,其格式参见\ref{subsec:translator};
\item |edition|:表示该析出文献所属会议录的版本,其格式参见\ref{subsec:bibfield-edition};
\item |address|:表示该析出文献所属会议录的出版地,其格式参见\ref{subsec:bibfield-address};
\item |publisher|:表示该析出文献所属会议录的出版者,其格式参见\ref{subsec:bibfield-publisher};
\item |citedate|:表示该析出文献的在线版本的引用日期,其格式参见\ref{subsec:bibfield-citedate};
\item |url|:表示该析出文献的在线版本的引用URL,其格式参见\ref{subsec:bibfield-url};
\item |doi|:表示该析出文献的电子版的DOI编号,其格式参见\ref{subsec:bibfield-doi};
\item |language|:表示该析出文献的语言,其格式参见\ref{subsec:bibfield-language};
\item 其他字段将不起作用。
\end{itemize}

\begin{note}
该会议录所属的丛书的标题不应出现在|booktitle|字段中,而应填写在|series|字段中。
该会议录所属的多卷书的卷号也不应出现在|booktitle|字段中,而应填写在|volume|字段中。
若该析出文献对应的文献项有|url|或|doi|字段,则其也必须有|citedate|字段。
若该析出文献的语言为中文,则必须将|language|字段设置为|zh|;否则可忽略|language|字段。
\end{note}

\subsubsection{例子}

\begin{verbatim}
@inproceedings{nonlinear1996,
  author={钟文发},
  title={非线性规划在可燃毒物配置中的应用},
  editor ={赵玮},
  booktitle={运筹学的理论与应用:中国运筹学会第五届大会论文集},
  address={西安},
  publisher={西安电子科技大学出版社},
  year={1996},
  pages={468-471},
  language={zh},
}

@inproceedings{fourney1971,
  author={Fourney M. E.},
  title={Advances in holographic photoelasticity},
  editor={{American Society of Mechanical Engineers, 
           Applied Mechanics Division}},
  booktitle={Symposium on Applications of Holography in Mechanics, 
             August 23--25, 1971, University of Southern California, 
             Los Angeles, California},
  address={New York},
  publisher={ASME},
  year={c1971},
  pages={17-38},
}

@inproceedings{aczel1998,
  author={Peter Aczel},
  title={On relating type theories and set theories},
  booktitle={Types `98, the proceedings of the 1998 workshop 
             on Types for Proofs and Programs},
  year={1998},
  editor={T. Altenkirch and W. Naraschewski and B. Reus},
  volume={1657},
  series={Lecture Notes in Computer Science},
  pages={1-18},
  address={Kloster Irsee},
  publisher={Springer},
}

@inproceedings{ariola1995,
  author={Zena M. Ariola and Matthias Felleisen 
          and John Maraist and Martin Odersky 
          and Philip Wadler},
  title={A Call-by-Need Lambda Calculus},
  booktitle={Proceedings of 22nd Annual {ACM {SIGACT}-{SIGPLAN}} 
             Symposium on Principles of Programming Languages
             ({POPL})},
  year={1995},
  pages={233-246},
}

@inproceedings{balaa2000,
  author={Antonia Balaa and Yves Bertot},
  title={Fix-point Equations for Well-Founded Recursion 
         in Type Theory},
  booktitle={Theorem Proving in Higher Order Logics},
  year={2000},
  editor={Mark Aagaard and John Harrison},
  volume={1869},
  series={Lecture Notes in Computer Science},
  pages={1-16},
  publisher={Springer-Verlag},
}

@inproceedings{barendregt1988,
  author={Henk Barendregt},
  title={Introduction to Lambda Calculus},
  booktitle={Aspen{\ae}s Workshop on Implementation of 
             Functional Languages, G{\"o}teborg},
  year={1988},
  address={G{\"o}teborg},
  publisher={Programming Methodology Group, University 
             of G{\"o}teborg and Chalmers
             University of Technology},
}

@inproceedings{basin1991,
  author={David A. Basin and Doulas J. Howe},
  title={Some Normalization Properties of Martin-L{\"o}f's 
         Type Theory and Applications},
  booktitle={International Conference on Theoretical 
         Aspects of Computer Software (TACS '91)},
  year={1991},
  volume={526},
  series={Lecture Notes in Computer Science},
  pages={475-494},
  publisher={Springer-Verlag},
}
\end{verbatim}

\subsection{期刊中的析出文献}\label{subsec:bibtype-article}

期刊中的析出文献即指期刊中的某篇论文或文章\cite{gbt7714-2005}。它所对应的{\BibTeX}文献项类
型为|article|;所对应的文献类型编码为|J|\cite{gbt3469-1983}。

\subsubsection{必需字段}

\begin{itemize}
\item |author|:表示该析出文献的作者,其格式参见\ref{subsec:bibfield-author};
\item |title|:表示该析出文献的标题,其格式参见\ref{subsec:bibfield-title};
\item |journal|:表示该析出文献所属期刊的名称,其格式参见\ref{subsec:bibfield-journal};
\item |year|:表示该析出文献所处的特定某期期刊的出版年,其格式参见\ref{subsec:bibfield-year}。
\end{itemize}

\subsubsection{可选字段}

\begin{itemize}
\item |volume|: 表示该析出文献所处的特定某期期刊的卷号,其格式参见\ref{subsec:bibfield-volume};
\item |number|: 表示该析出文献所处的特定某期期刊的期号,其格式参见\ref{subsec:bibfield-number};
\item |pages|表示该析出文献在其所处的特定某期期刊中的页码或页码范围,其格式参见\ref{subsec:bibfield-pages};
\item |citedate|:表示该析出文献的在线版本的引用日期,其格式参见\ref{subsec:bibfield-citedate};
\item |url|:表示该析出文献的在线版本的引用URL,其格式参见\ref{subsec:bibfield-url};
\item |doi|:表示该析出文献的电子版的DOI编号,其格式参见\ref{subsec:bibfield-doi};
\item |language|:表示该析出文献的语言,其格式参见\ref{subsec:bibfield-language};
\item 其他字段将不起作用。
\end{itemize}

\begin{note}
若该析出文献对应的文献项有|url|或|doi|字段,则其也必须有|citedate|字段。
若该析出文献的语言为中文,则必须将|language|字段设置为|zh|;否则可忽略|language|字段。
\end{note}

\subsubsection{例子}

\begin{verbatim}
@article{kanamori1998,
  author={Kanamori, H.},
  title={Shaking without quaking},
  journal={Science},
  year={1998},
  volume={279},
  number={5359},
  pages={2063-2064},
}

@article{caplan1993,
  author={Caplan, P.},
  title={Cataloging internet resources},
  journal={The Public Access Computer Systems Review},
  year={1993},
  volume={4},
  number={2},
  pages={61-66},
}

@article{marais1992,
  author={des Marais, D. J. and Strauss, H. 
          and Summons, R. E. and Other Authors},
  title={Carbon isotope evidence for the stepwise 
         oxidation of the Proterozoic environment},
  journal={Nature},
  year={1992},
  volume={359},
  pages={605-609},
}

@article{hewitt1984,
  author={Hewitt, J. A.},
  title={Technical services in 1983},
  journal={Library Resource Services},
  year={1984},
  volume={28},
  number={3},
  pages={205-218},
}

@article{christine1998,
  author={Christine, M.},
  title={Plant physiology: plant biology in the Genome Era},
  journal={Science},
  year={1998},
  volume={281},
  pages={331-332},
  url={http://www.sciencemag.org/cgi/collection/anatmorp},
  citedate={1998-09-23},
}

@article{lixiaodong1999,
  author={李晓东 and 张庆红 and 叶瑾琳},
  title={气候学研究的若干理论问题},
  journal={北京大学学报: 自然科学版},
  year={1999},
  volume={35},
  number={1},
  pages={101-106},
  language={zh},
}

@article{liuwu1999,
  author={刘武 and 郑良 and 姜础},
  title={元谋古猿牙齿测量数据的统计分析及其在分类研究上的意义},
  journal={科学通报},
  year={1999},
  volume={44},
  pages={2481-2488},
  language={zh},
}

@article{liruimu2000,
   author={李炳穆},
   title={理想的图书馆员和信息专家的素质与形象},
   journal={图书情报工作},
   year={2000},
   number={2},
   pages={5-8},
   language={zh},
}

@article{moshaoqiang1999,
  author={莫少强},
  title={数字式中文全文文献格式的设计与研究},
  journal={情报学报},
  year={1999},
  volume={18},
  number={4},
  pages={1-6},
  url={http://periodical.wanfangdata.com.cn/periodical/qbxb9904/990407.htm},
  citedate={2001-07-08},
  language={zh},
}

@article{taorenji1984,
  author={陶仁骥},
  title={密码学与数学},
  journal={自然杂志},
  year={1984},
  volume={7},
  number={7},
  pages={527},
  language={zh},
}

@article{yazhou1978,
  author={{亚洲地质图编目组}},
  title={亚洲地层与地质历史概述},
  journal={地质学报},
  year={1978},
  volume={3},
  pages={104-208},
  language={zh},
}

@article{jiangxiangdong1999,
  author={江向东},
  title={互联网环境下的信息处理与图书管理系统解决方案},
  journal={情报学报},
  year={1999},
  volume={18},
  number={2},
  pages={4},
  url={http://www.chinainfo.gov.cn/periodical/qbxb99/qbxb990203},
  citedate={2000-01-18},
  language={zh},
}
\end{verbatim}

\subsection{新闻报道}\label{subsec:bibtype-news}

新闻报道是指刊登在报纸上的报道。它所对应的{\BibTeX}文献项类型为|news|。
它所对应的文献类型编码为|N|\cite{gbt3469-1983}。

|news|类型不是{\BibTeX}的标准类型,而是{\njuthesis}的扩展。

\subsubsection{必需字段}

\begin{itemize}
\item |author|:表示该新闻报道的作者,其格式参见\ref{subsec:bibfield-author};
\item |title|:表示该新闻报道的标题,其格式参见\ref{subsec:bibfield-title};
\item |journal|:表示刊登该新闻报道的报刊的名称,其格式参见\ref{subsec:bibfield-journal};
\item |date|:表示刊登该新闻报道的特定某期报刊的发行日期,其格式参见\ref{subsec:bibfield-date};
\end{itemize}

\subsubsection{可选字段}

\begin{itemize}
\item |number|: 表示刊登该新闻报道的特定某期报刊的期号,其格式参见\ref{subsec:bibfield-number}。
\item |citedate|:表示该新闻报道的在线版本引用日期,其格式参见\ref{subsec:bibfield-citedate};
\item |url|:表示该新闻报道的在线版本的引用URL,其格式参见\ref{subsec:bibfield-url};
\item |doi|:表示该新闻报道的电子版的DOI编号,其格式参见\ref{subsec:bibfield-doi};
\item |language|:表示该新闻报道的语言,其格式参见\ref{subsec:bibfield-language};
\item 其他字段将不起作用。
\end{itemize}

\begin{note}
若该新闻报道对应的文献项有|url|或|doi|字段,则其也必须有|citedate|字段。
若该新闻报道的语言为中文,则必须将|language|字段设置为|zh|;否则可忽略|language|字段。
\end{note}

\subsubsection{例子}

\begin{verbatim}
@news{ding2000,
  author={丁文祥},
  title={数字革命与竞争国际化},
  journal={中国青年报},
  date={2000-11-20},
  number={15},
  language={zh},
}

@news{zhang2000,
  author={张田勤},
  title={罪犯DNA库与生命伦理学计划},
  journal={大众科技报},
  date={2000-11-12},
  language={zh},
}

@news{fu2000,
  author={傅刚 and 赵承 and 李佳路},
  title={大风沙过后的思考},
  journal={北京青年报},
  date={2000-04-12},
  number={14},
  url={http://www.bjyouth.com.cn/Bgb/20000412/GB/4216%5ED0412B1401.htm},
  citedate={2005-07-12},
  language={zh},
}
\end{verbatim}

%%%%%%%%%%%%%%%%%%%%%%%%%%%%%%%%%%%%%%%%%%%%%%%%%%%%%%%%%%%%%%%%%%%%%%%%%%%%%%%

\section{文献项字段}\label{sec:bib-field}

本节将描述{\njuthesis}文档类所支持的文献项字段。

\begin{note}
因为要遵循\std{GB/T 7714-2005}规范\cite{gbt7714-2005},所以本节所描述的文献项字段
可能和标准{\BibTeX}的文献项字段有所差别。
\end{note}

\subsection{author}\label{subsec:bibfield-author}

|author|字段表示文献的作者。

|author|字段可以有多个作者名。多个作者姓名之间\emph{必须}用英文词``and''隔开。根据
\cite{gbt7714-2005}的规定,{\BibTeX}会自动只使用前三个作者姓名进行排版,若还有多余作者,
会在作者姓名之后加``, 等''(对于中文文献)或``, et al''(对于外文文献)。

外文作者名建议按照\cite[157]{lamport1994latex}中的要求填写,{\BibTeX}系统会自动将其
转换为符合\std{GB/T 7714-2005}规范\cite{gbt7714-2005}的格式。具体可以参见
\ref{sec:names}。例如:
\begin{itemize}
\item |author = {Rogers, Jr., Hartley}|
\item |author = {Robert L. Constable and Stuart F. Allen}|
\end{itemize}

中文作者名采用姓在前名在后的方式填写,姓和名之间不需加空格。多个作者姓名之间也\emph{必须}用
英文词``and''隔开。例如:
\begin{itemize}
\item |author = {赵凯华}|
\item |author = {赵凯华 and 罗蔚茵 and 张三 and 李四 and 王五}|
\end{itemize}

中国作者姓名的汉语拼音的拼写执行\std{GB/T 16159—2012}的规定\cite{gbt16159-2012},
名字不能缩写。例如:``Zheng Guangmei''或``ZHENG Guangmei''。为了防止{\BibTeX}将
中国作者姓名的汉语拼音缩写,应该在该作者名两端用``{}''将其括起来。例如:
\begin{itemize}
\item |author = {{ZHAO Kaihua}}|
\item |author = {{ZHAO Kaihua} and {LUO Weiyin} and {ZHANG Sang}}|
\end{itemize}

对于中文文献,若其是外文文献的翻译版,应将其作者名写成中译名。

外国作者的中译名可以只著录其姓;同姓不同名的外国作者,其中译名不仅要著录其姓,还需著
录其名。外国作者的名可以用缩写字母,缩写名后可省略缩写点。翻译者的姓名应使用|translator|
字段指定。例如:
\begin{itemize}
\item |author = {马克思 and 恩格斯}|
\item |author = {昂温, G. and 昂温, P. S.}|
\end{itemize}

凡是对文献负责的机关团体名称通常根据著录信息源著录。用拉丁文书写的机关团体名称应由上至下分级著录 

如果文献的主要责任者是某个组织,例如公司或政府部门,其对应的文献项的|author|字段
应该填写该组织的名称。机关团体名称应由上至下分级著录。为了防止{\BibTeX}将其识
别为一个姓名,应该在组织名称两端用``{}''将其括起来。例如:
\begin{itemize}
\item |author = {{American Chemical Society}}|
\item |author = {{Stanford University, Department of Civil Engineering}}|
\item |author = {{中国科学院数学研究所}}|
\item |author = {{广西壮族自治区林业厅}}|
\end{itemize}


\subsection{editor}\label{subsec:bibfield-editor}

|editor|字段表示文献的编辑者。其格式的具体要求与|author|字段类似。

|editor|字段和|author|字段的区别在于:|author|是所引用的文献的主要负责人,
|editor|则是所引用文献本身,或其所在的合集的编辑。例如:
\begin{itemize}
\item 如果所文献是一本书,但每个章节由不同的作者所写,而整本书由某个编辑者编辑。此时,
  \begin{itemize}
  \item 如果论文中引用的是全书,则应该将该文献作为|book|类型,其|editor|字段为编辑
    者,|author|字段不填。
  \item 如果论文中引用的是全书中由某个作者写的某个章节,则应该将文献作为
    |inbook|或|incollection|类型(取决于该章节是否有自己的标题,详见
    \ref{sec:bib-type}的解释),其|author|字段为该章节作者,|editor|字段为
    专著的编辑者。
  \end{itemize}
\item 如果文献是会议论文集,此时:
  \begin{itemize}
  \item 如果论文引用的是论文集全集,则应该将文献作为|proceedings|类型,其|editor|字段
    为该论文集的编辑,|author|字段不填;
  \item 如果论文引用的是论文集中的某篇文章,则应该将文献作为|inproceedings|类型,
    其|author|字段为该文章作者,|editor|字段为该论文集的编辑者。
  \end{itemize}
\end{itemize}

关于|editor|和|author|字段的详细解释,请参见\ref{sec:bib-type}。

\subsection{translator}\label{subsec:bibfield-translator}

|translator|字段表示文献的翻译者。其格式的具体要求与|author|字段类似。

例如:
\begin{itemize}
\item |translator = {张三}|
\item |translator = {张三 and 李四}|
\item |translator = {Bill Gates}|
\item |translator = {Bill Gates and Bill Clinton}|
\end{itemize}

\subsection{title}\label{subsec:bibfield-title}

|title|字段表示文献的标题。如果该文献是析出文献(即某个专著中的某一章、某一节或某篇文章),
则|title|字段应该是被引用的析出文章或章节的标题;否则,|title|字段应该是被引用的文献的标题。

若被引用的文献的标题有副标题或其他标题,应将其与主标题用冒号隔开。

例如:
\begin{itemize}
\item |title = {Introduction to Algorithm}|
\item |title = {The Art of Computer Programming: Volume 3, Sorting and Searching}|,
\item |title = {马克思恩格斯全集: 第44卷}|
\end{itemize}

\subsection{booktitle}\label{subsec:bibfield-booktitle}

|booktitle|字段表示析出文献所在的专著的标题。其要求与|title|字段类似,具体
参见\ref{subsec:bibfield-title}。

\subsection{series}\label{subsec:bibfield-series}

|series|字段表示文献所属丛书的标题。其要求与|title|字段类似,具体
参见\ref{subsec:bibfield-title}。

\subsection{volume}\label{subsec:bibfield-volume}

|volume|字段表示文献的卷号。卷号用一个阿拉伯数字表示。例如:
\begin{itemize}
\item |volume = {13}|
\end{itemize}

在|periodical|类型文献中,|volume|字段可用于表示所引用的一系列连续期刊的卷号的范围
(参见\ref{subsec:bibtype-periodical})。此时用一个短横``-''连接两个阿拉伯数字表示
的卷号,前者为起始期刊的卷号,后者为结束期刊的卷号。例如:
\begin{itemize}
\item 卷号范围:|volume = {13-25}|
\end{itemize}

\subsection{number}\label{subsec:bibfield-number}

|number|字段表示文献的期号。期号用一个阿拉伯数字表示。例如:
\begin{itemize}
\item |number = {13}|
\end{itemize}

在|periodical|类型文献中,|number|字段可用于表示所引用的一系列连续期刊的期号的范围
(参见\ref{subsec:bibtype-periodical})。此时用一个短横``-''连接两个阿拉伯数字表示
的期号,前者为起始期刊的期号,后者为结束期刊的期号。例如:
\begin{itemize}
\item 期号范围:|number = {13-25}|
\end{itemize}


\subsection{edition}\label{subsec:bibfield-edition}

|edition|字段表示专著的版本号。

该字段值有两种可取类型:
\begin{itemize}
\item 如果该字段的值是一个整数,则该整数表示专著的版本号。{\BibTeX}会使用类似``3版''或
``3rd ed''这样的方式排版版本号;
\item 如果专著是第一版,则无需此字段;
\item 如果该字段不是一个整数,则被当做一个字符串。{\BibTeX}会直接把该字符串排版在版本号的位置,
不做任何处理。
\item 如果是用英文表示的版本,通常应该以`` ed''结尾;如果是中文表示的版本,通常应该
以``xx版''的形式,序数前面无需加``第''字。
\end{itemize}

例如:
\begin{itemize}
\item 直接用整数表示版本号:|edition = {2}|
\item 以年代表示版本号:|edition = {1994 ed}|
\item 以年代表示版本号:|edition = {1994版}|
\item 其他版本号:|edition = {石印本}|
\end{itemize}

\subsection{address}\label{subsec:bibfield-address}

|address|字段表示文献的出版地。

出版地著录出版者所在地的城市名称。对同名异地或不为人们熟悉的城市名,应在城市名后
附省名、州名或国名等限定语。例如:
\begin{itemize}
\item |address = {北京}|
\item |address = {London}|
\item |addrees = {Cambridge, Eng}|
\item |address = {Cambridge, Mas}|
\end{itemize}

文献中载有多个出版地,只著录第一个或处于显要位置的出版地。例如:
\begin{itemize}
\item ``科学出版杜 北京 上海 2000''应著录为:|address = {北京}|, 
  |publisher = {科学出版杜}|, |year = {2000}|
\item ``Buterworths London Boston Sydney Welington Durban Toronto 1978''应著
  录为:|address = {London}|, |publisher = {Buterworths}|, |year = {1978}|
\end{itemize}

若该类型的文献需要出版地,但其{\BibTeX}项中缺少|address|字段,则{\BibTeX}排版后
会用``[S.l.]'' (外文文献)或``[出版地不详]''(中文文献)替代。

\subsection{publisher}\label{subsec:bibfield-publisher}

|publisher|字段表示文献的出版者。

出版者可以按著录信息源所载的形式著录,也可以按国际公认的简化形式或缩写形式著录。
例如:
\begin{itemize}
\item ``科学出版社''可著录为:|publisher = {科学出版社}|
\item ``Elsevier Science Publishers''可著录为:|publisher = {Elsevier Science Publishers}|
\item ``International Rice Research Institute''可著录为:|publisher = {IRRI}|
\item ``John Wiley and Sons Ltd''可著录为:|publisher = {Wiley}|
\end{itemize}

著录信息源载有多个出版者,只著录第一个或处于显要位置的出版者。例如:
\begin{itemize}
\item ``American Library Asociation/Chicago Canadian Library Asociation/Otawa 1978''应
    著录为:|address={Chicago}|, |publisher={American Library Asociation}|, |year={1978}|
\end{itemize}

若该类型的文献需要出版者,但其{\BibTeX}项中缺少|publisher|字段,则{\BibTeX}排版后
会用``[s.n.]'' (外文文献)或``[出版者不详]''(中文文献)替代。

\subsection{school}\label{subsec:bibfield-school}

|school|字段表示文献所申请学位的颁发机构。

\subsection{institution}\label{subsec:bibfield-institution}

|institution|表示文献作者所属机构。

\subsection{year}\label{subsec:bibfield-year}

|year|字段表示文献的出版年,或文献的出版年的范围。

出版年采用公元纪年,并用阿拉伯数字著录。如有其他纪年形式时,将原有的纪年形式置于圆括号内。
例如:
\begin{itemize}
\item |year = {1947(民国三十六年)}|
\item |year = {1705(康熙 四十四年)}|
\end{itemize}

出版年无法确定时,可依次选用版权年、印刷年、估计的出版年。估计的出版年需置于方括号内。
例如:
\begin{itemize}
\item 版权年:|year = {c1998}|
\item 印刷年:|year = 1995印刷|
\item 估计的出版年:|year = {[1936]}|
\end{itemize}

在|periodical|类型文献中,|year|字段可用于表示所引用的一系列连续期刊的出版年的范围
(参见\ref{subsec:bibtype-periodical})。此时用一个短横``-''连接两个阿拉伯数字表示
的年份,前者为起始期刊的出版年,后者为结束期刊的出版年。例如:
\begin{itemize}
\item 出版年范围:|year = {1957-1990}|
\end{itemize}


\subsection{pages}\label{subsec:bibfield-pages}

|pages|字段表示析出文献所在的页码或页码范围。页码用阿拉伯数字表示。页码范围用两个阿拉伯数字
分别表示起始页页码和结束页页码,两者之间用一个短横隔开。注意,页码字段不要包含诸如``p.'',
``page'',``页''之类的描述字符。例如:
\begin{itemize}
\item |pages = {128}|
\item |pages = {128--234}|
\end{itemize}

\subsection{modifydate}\label{subsec:bibfield-modifydate}

|modifydate|表示在线版文献的最后修改日期。此字段不是标准的{\BibTeX}字段,而是{\njuthesis}
的扩展。

最后修改日期必须以阿拉伯数字表示年、月、日。“年”用四位阿拉伯数字表示;“月”用两位阿拉伯数字表示,
不足补零;“日”用两位阿拉伯数字表示,不足补零。三个部分之间用一个短横隔开。
例如:
\begin{itemize}
\item |modifydate = {2013-08-30}|
\end{itemize}

\subsection{citedate}\label{subsec:bibfield-citedate}

|citedate|表示在线版文献的引用日期。此字段不是标准的{\BibTeX}字段,而是{\njuthesis}
的扩展。

引用日期必须以阿拉伯数字表示年、月、日。“年”用四位阿拉伯数字表示;“月”用两位阿拉伯数字表示,
不足补零;“日”用两位阿拉伯数字表示,不足补零。三个部分之间用一个短横隔开。
例如:
\begin{itemize}
\item |citedate = {2013-08-30}|
\end{itemize}

如果文献项有|url|或|doi|字段,则也应有相应的|citedate|字段。

\subsection{url}\label{subsec:bibfield-url}

|url|字段表示文献的在线URL地址。

如果某文献项有|url|字段,则该文献被认为是在线文献。{\BibTeX}处理后,会在其文献类型后增
加``/OL'',表示其属于在线文献。

如果某文献项有|url|字段,则它也应该有对应的|citedate|字段。

例如:
\begin{itemize}
\item |url = {http://www.di.ens.fr/users/longo/download.html}|
\item |url = {http://www.bjyouth.com.cn/20000412/GB/4216%5ED0412B1401.htm}|
\end{itemize}

\subsection{doi}\label{subsec:bibfield-doi}

|doi|字段表示文献的``数字对象标识号(Digital Object Identifier)''。例如:
\begin{itemize}
\item |doi = {10.1007/s00223-003-0070-0}|
\end{itemize}

数字对象识别号是一套识别数字资源的机制,涵括的对象有视频、报告或书籍等等。它既有一套为资源命名的
机制,也有一套将识别号解析为具体地址的协议。

发展DOI的动机在于补充URI之不足,因为一方面URI指涉的URL经常变动,另一方面,URI表达的其实是资源
所在地(即网址),而非数字资源本身的信息。DOI能克服这两个问题。

一个DOI识别号经过解析后,可以连至一个或更多的数据。但识别号本身与解析后导向的数据并不相干,也可
能发生无法取得全部数据,只能得到相关出版品信息的情形。DOI的解析协议见诸RFC 3652,RFC 3651描
述命名机制,RFC 3650描述的则是其架构。DOI通过Handle系统解析识别号,但实际应用上大多是通过网
站解析.例如连进网址\url{http://dx.doi.org/10.1007/s00223-003-0070-0},就能看到对应识
别号\texttt{10.1007/s00223-003-0070-0}的论文信息或全文。

如果某文献项只有|doi|字段但没有|url|字段,该文献也被认为是在线文献。{\BibTeX}处理后,会在
其文献类型后增加``/OL'',表示其属于在线文献;同时也会自动根据DOI的解析网址前缀
\url{http://dx.doi.org/}合成其在线URL地址。

如果某文献项有|doi|字段,则它也应该有对应的|citedate|字段。


\subsection{media}\label{subsec:bibfield-media}

|media|字段表示引用文献的载体。此字段不是标准的{\BibTeX}字段,而是{\njuthesis}的扩展。

若引用文献对应的{\BibTeX}项有|url|或|doi|字段,此字段将被忽略,且该文献的载体将
被自动设置为|OL|,表示联机网络。

若引用文献对应的{\BibTeX}项没有|media|字段,则默认该文献的载体为一般纸质媒体。

否则,引用文献对应的{\BibTeX}项的|media|字段应该按照表\ref{tab:media-code}填写。

\begin{table}
\centering
\begin{tabular}{|C{5cm}|C{5cm}|}
\toprule
  \textbf{载体类型} & \textbf{标识代码} \\
\midrule
  磁带(magnetic tape) & MT \\
\hline
  磁盘(disk)  & DK \\
\hline
  光盘(CD-ROM) & CD \\
\hline
  联机网络(online) & OL \\
\bottomrule
\end{tabular}
\caption{电子文献载体和标志编码}\label{tab:media-code}
\end{table}


\subsection{language}\label{subsec:bibfield-language}

|language|字段用于说明该文献项对应的文献所使用的语言。例如:
\begin{itemize}
\item |language = {zh}|
\item |language = {en}|
\end{itemize}

如果是外文文献,此字段可忽略;如果是中文文献,此字段\emph{必须}填写|zh|。

{\BibTeX}需要根据此字段判别文献的语言,对于中文文献和外文文献采用不同的排版方式。

选择文献语言的基本原则是,若该文献本身是中文的(即使是翻译版),其对应项的语言就应该选|zh|,
其标题、作者等也应该尽量以中文填写。

%%%%%%%%%%%%%%%%%%%%%%%%%%%%%%%%%%%%%%%%%%%%%%%%%%%%%%%%%%%%%%%%%%%%%%%%%%%%%%%


%%%%%%%%%%%%%%%%%%%%%%%%%%%%%%%%%%%%%%%%%%%%%%%%%%%%%%%%%%%%%%%%%%%%%%%%%%%%%%%
%%  
%% 文档类 NJU-Thesis 用户手册
%%
%% 作者:胡海星,starfish (at) gmail (dot) com
%% 项目主页: https://github.com/Haixing-Hu/nju-thesis
%%
%% This file may be distributed and/or modified under the conditions of the
%% LaTeX Project Public License, either version 1.2 of this license or (at your
%% option) any later version. The latest version of this license is in:
%%
%% http://www.latex-project.org/lppl.txt
%%
%% and version 1.2 or later is part of all distributions of LaTeX version
%% 1999/12/01 or later.
%%
%%%%%%%%%%%%%%%%%%%%%%%%%%%%%%%%%%%%%%%%%%%%%%%%%%%%%%%%%%%%%%%%%%%%%%%%%%%%%%%

\chapter{学位论文的编译}

\section{简介}

\section{编译命令}

\section{常见错误及解决方法}

\subsection{中文文献中出现英文}

比如出现``张三,李四,王五,et al''或``Translated by 张三''等,这是因为该文献对应的
{\BibTeX}项的|language|字段未被设置为|zh|。

\subsection{参考文献项目中某些项空白}

仔细检查一下该项在{\BibTeX}中每个字段之后是否有逗号``,''。

%%%%%%%%%%%%%%%%%%%%%%%%%%%%%%%%%%%%%%%%%%%%%%%%%%%%%%%%%%%%%%%%%%%%%%%%%%%%%%%
%%  
%% 文档类 NJU-Thesis 用户手册
%%
%% 作者:胡海星,starfish (at) gmail (dot) com
%% 项目主页: https://github.com/Haixing-Hu/nju-thesis
%%
%% This file may be distributed and/or modified under the conditions of the
%% LaTeX Project Public License, either version 1.2 of this license or (at your
%% option) any later version. The latest version of this license is in:
%%
%% http://www.latex-project.org/lppl.txt
%%
%% and version 1.2 or later is part of all distributions of LaTeX version
%% 1999/12/01 or later.
%%
%%%%%%%%%%%%%%%%%%%%%%%%%%%%%%%%%%%%%%%%%%%%%%%%%%%%%%%%%%%%%%%%%%%%%%%%%%%%%%%

\chapter{结论}


% 致谢章节
%%%%%%%%%%%%%%%%%%%%%%%%%%%%%%%%%%%%%%%%%%%%%%%%%%%%%%%%%%%%%%%%%%%%%%%%%%%%%%%
%%  
%% 文档类 NJU-Thesis 用户手册
%%
%% 作者:胡海星,starfish (at) gmail (dot) com
%% 项目主页: https://github.com/Haixing-Hu/nju-thesis
%%
%% This file may be distributed and/or modified under the conditions of the
%% LaTeX Project Public License, either version 1.2 of this license or (at your
%% option) any later version. The latest version of this license is in:
%%
%% http://www.latex-project.org/lppl.txt
%%
%% and version 1.2 or later is part of all distributions of LaTeX version
%% 1999/12/01 or later.
%%
%%%%%%%%%%%%%%%%%%%%%%%%%%%%%%%%%%%%%%%%%%%%%%%%%%%%%%%%%%%%%%%%%%%%%%%%%%%%%%%

\begin{acknowledgement}


\end{acknowledgement}

% 附录
\appendix

\chapter{南京大学博士(硕士)学位论文编写格式规定(试行)}\label{chap:njureq}

\section{适用范围}

本规定适用于博士学位论文编写,硕士学位论文编写应参照执行。

\section{引用标准}

GB7713科学技术报告、学位论文和学术论文的编写格式。

GB7714文后参考文献著录规则。

\section{印制要求}

论文必须用白色纸印刷,并用A4(210mm×297mm)标准大小的白纸。纸的四周应留足空白
边缘,上方和左侧应空边25mm以上,下方和右侧应空边20mm以上。除前置部分外,其它
部分双面印刷。

论文装订不要用铁钉,以便长期存档和收藏。

论文封面与封底之间的中缝(书脊)必须有论文题目、作者和学校名。

\section{编写格式}

论文由前置部分、主体部分、附录部分(必要时)、结尾部分(必要时)组成。

前置部分包括封面,题名页,声明及说明,前言,摘要(中、英文),关键词,目次页,
插图和附表清单(必要时),符号、标志、缩略词、首字母缩写、单位、术语、名词解释
表(必要时)。

主体部分包括绪论(作为正文第一章)、正文、结论、致谢、参考文献表。

附录部分包括必要的各种附录。

结尾部分包括索引和封底。

\section{前置部分}

\subsection{封面(博士论文国图版用)}

封面是论文的外表面,提供应有的信息,并起保护作用。

封面上应包括下列内容:
\begin{enumerate}
\item 分类号  在左上角注明分类号,便于信息交换和处理。一般应注明《中国图书资
  料分类法》的类号,同时应注明《国际十进分类法UDC》的类号;
\item 密级  在右上角注明密级;
\item “博士学位论文”用大号字标明;
\item 题名和副题名   用大号字标明;
\item 作者姓名;
\item 学科专业名称;
\item 研究方向;
\item 导师姓名,职称;
\item 日期包括论文提交日期和答辩日期;
\item 学位授予单位。
\end{enumerate}

\subsection{题名}

题名是以最恰当、最简明的词语反映论文中最重要的特定内容的逻辑组合。

题名所用每一词语必须考虑到有助于选定关键词和编写题录、索引等二次文献可以提供
检索的特定实用信息。

题名应避免使用不常见的缩略词、首字母缩写字、字符、代号和公式等。

题名一般不宜超过20字。

论文应有外文题名,外文题名一般不宜超过10个实词。

可以有副题名。

题名在整本论文中不同地方出现时,应完全相同。

\subsection{前言}

前言是作者对本论文基本特征的简介,如论文背景、主旨、目的、意义等并简述本论文
的创新性成果。

\subsection{摘要}

摘要是论文内容不加注释和评论的简单陈述。

论文应有中、英文摘要,中、英文摘要内容应相同。

摘要应具有独立性和自含性,即不阅读论文的全文,便能获得必要的信息,摘要
中有数据、有结论,是一篇完整的短文,可以独立使用,可以引用,可以用于推广。摘
要的内容应包括与论文同等量的主要信息,供读者确定有无必要阅读全文,也供文摘等
二次文献引用。摘要的重点是成果和结论。

中文摘要一般在1500字,英文摘要不宜超过1500实词。

摘要中不用图、表、化学结构式、非公知公用的符号和术语。

\subsection{关键词}

关键词是为了文献标引工作从论文中选取出来用于表示全文主题内容信息款目的单词或
术语。

每篇论文选取3-8个词作为关键词,以显著的字符另起一行,排在摘要的左下方。在英
文摘要的左下方应标注与中文对应的英文关键词。

\subsection{目次页}

目次页由论文的章、节、附录等的序号、名称和页码组成,另页排在摘要的后面。

\subsection{插图和附表清单}

论文中如图表较多,可以分别列出清单并置于目次页之后。

图的清单应有序号、图题和页码。表的清单应有序号、表题和页码。

符号、标志、缩略词、首字母缩写、计量单位、名词、术语等的注释表符号、标志、缩略词、
首字母缩写、计量单位、名词、术语等的注释说明汇集表,应置于图表清单之后。

\section{主体部分}

\subsection{格式}

主体部分由绪论开始,以结论结束。主体部分必须由另页右页开始。每一章必须另页开
始。全部论文章、节、目的格式和版面安排要划一,层次清楚。

\subsection{序号}

论文的章可以写成:第一章。节及节以下均用阿拉伯数字编排序号,如
1.1,1.1.1等。

论文中的图、表、附注、参考文献、公式、算式等一律用阿拉伯数字分别分章依序连续编排
序号。其标注形式应便于互相区别,一般用下例:图1.2;表2.3;附注1);文献[4];式
  (6.3)等。

论文一律用阿拉伯数字连续编页码。页码由首页开始,作为第1页,并为右页另页。封页、
封二、封三和封底不编入页码,应为题名页、前言、目次页等前置部分单独编排页码。页码
必须标注在每页的相同位置,便于识别。

附录依序用大写正体A、B、C、$\cdots$编序号,如:附录A。附录中的图、表、式、参考文
献等另行编序号,与正文分开,也一律用阿拉伯数字编码,但在数码前题以附条序码,如图
A.1;表B.2;式(B.3);文献[A.5]等。

\subsection{绪论}

绪论(综述):简要说明研究工作的目的、范围、相关领域的前人工作和知识空白、理
论基础和分析,研究设想、研究方法和实验设计、预期结果和意义等。一般在教科书中
有的知识,在绪论中不必赘述。

绪论的内容应包括论文研究方向相关领域的最新进展、对有关进展和问题的评价、本论
文研究的命题和技术路线等;绪论应表明博士生对研究方向相关的学科领域有系统深入
的了解,论文具有先进性和前沿性;

为了反映出作者确已掌握了坚实的基础理论和系统的专门知识,具有开阔的科学视野,对研
究方案作了充分论证,绪论应单独成章,列为第一章,绪论的篇幅应达$1\sim 2$万字,不
得少于$1$万字;绪论引用的文献应在$100$篇以上,其中外文文献不少于$60\%$;引用文献
应按正文中引用的先后排列。

\subsection{正文}

论文的正文是核心部分,占主要篇幅。正文必须实事求是,客观真切,准确完备,合乎
逻辑,层次分明,简便可读。

正文的每一章(除绪论外)应有小结,在小结中应明确阐明作者在本章中所做的工作,特
别是创新性成果。凡本论文要用的基础性内容或他人的成果不应单独成章,也不应作过
多的阐述,一般只引结论、使用条件等,不作推导。

\subsubsection{图}

图包括曲线图、构造图、示意图、图解、框图、流程图、记录图、布置图、地图、照片
、图版等。

图应具有“自明性”,即只看图、图题和图例,不阅读正文,就可以理解图意。

图应编排序号。每一图应有简短确切的图题,连同图号置于图下。必要时,应将图上的
符号、标记、代码,以及实验条件等,用最简练的文字,横排于图题下方,作为图例说
明。

曲线图的纵、横坐标必须标注“量、标准规定符号、单位”。此三者只有在不必要标明
(如无量纲等)的情况下方可省略。坐标上标注的量的符号和缩略词必须与正文一致。

照片图要求主题和主要显示部分的轮廓鲜明,便于制版。如用放大缩小的复制品,必须
清晰,反差适中。照片上应该有表示目的物尺寸的标度。

\subsubsection{表}

表的编排,一般是内容和测试项目由左至右横读,数据依序竖排。表应有自明性。

表应编排序号。

每一表应有简短确切的表题,连同标号置于表上。必要时,应将表中的符号、标记、代
码,以及需要说明事项,以最简练的文字,横排于表题下,作为表注,也可以附注于表
下。表内附注的序号宜用小号阿拉伯数字并加圆括号置于被标注对象的右上角,如:
xxx${}^{1)}$;不宜用“*”,以免与数学上共轭和物质转移的符号相混。

表的各栏均应标明“量或测试项目、标准规定符号、单位”。只有在无必要标注的情况下
方可省略。表中的缩略词和符号,必须与正文中一致。

表内同一栏的数字必须上下对齐。表内不宜用“同上”,“同左”和类似词,一律填入具体数字
或文字。表内“空白”代表未测或无此项,“-”或“\textellipsis”(因“-”可能与代表阴性
  反应相混)代表未发现,“0”代表实测结果确为零。

如数据已绘成曲线图,可不再列表。

\subsubsection{数学、物理和化学式}

正文中的公式、算式或方程式等应编排序号,序号标注于该式所在行(当有续行时,应
标注于最后一行)的最右边。

较长的式,另行居中横排。如式必须转行时,只能在$+$,$-$,$\times$,$\div$,$<$,
$>$处转行。上下式尽可能在等号“$=$”处对齐。

小数点用“$.$”表示。大于$999$的整数和多于三位数的小数,一律用半个阿拉伯数字符的小
间隔分开,不用千位撇。对于纯小数应将$0$列于小数点之前。

示例:应该写成$94\ 652.023\ 567$和$0.314\ 325$, 不应写成$94,652.023,567$和
$.314,325$。

应注意区别各种字符,如:拉丁文、希腊文、俄文、德文花体、草体;罗马数字和阿拉伯数
字;字符的正斜体、黑白体、大小写、上下脚标(特别是多层次,如“三踏步”)、上下偏差
等。

\subsubsection{计量单位}

报告、论文必须采用国务院发布的《中华人民共和国法定计量单位》,并遵照《中华人
民共和国法定计量单位使用方法》执行。使用各种量、单位和符号,必须遵循附录B所
列国家标准的规定执行。单位名称和符号的书写方式一律采用国际通用符号。

\subsubsection{符号和缩略词}

符号和缩略词应遵照国家标准的有关规定执行。如无标准可循,可采纳本学科或本专业
的权威性机构或学术团体所公布的规定;也可以采用全国自然科学名词审定委员会编印
的各学科词汇的用词。如不得不引用某些不是公知公用的、且又不易为同行读者所理解
的、或系作者自定的符号、记号、缩略词、首字母缩写字等时,均应在第一次出现时一
一加以说明,给以明确的定义。

\subsection{结论}

报告、论文的结论是最终的、总体的结论,不是正文中各段的小结的简单重复。结论应
该准确、完整、明确、精炼。在结论中要清楚地阐明论文中有那些自己完成的成果,特
别是创新性成果;

如果不可能导出应有的结论,也可以没有结论而进行必要的讨论。可以在结论或讨论中
提出建议、研究设想、仪器设备改进意见、尚待解决的问题等。

\subsection{致谢}

可以在正文后对下列方面致谢:

\begin{itemize}
\item 国家科学基金、资助研究工作的奖学金基金、合作单位、资助或支持的企业、组织或个
人;
\item 协助完成研究工作和提供便利条件的组织或个人;
\item 在研究工作中提出建议和提供帮助的人;
\item 给予转载和引用权的资料、图片、文献、研究思想和设想的所有者;
\item 其他应感谢的组织或个人。
\end{itemize}

\subsection{参考文献表}

\subsubsection{专著著录格式}

主要责任者,其他责任者,书名,版本,出版地:出版者,出版年

例:1. 刘少奇,论共产党员的修养,修订2版,北京:人民出版社,1962

\subsubsection{连续出版物中析出的文献著录格式}

析出文献责任者,析出文献其他责任者,析出题名,原文献题名,版本:文献中的位置。

例:2. 李四光,地壳构造与地壳运动,中国科学,1973 (4):400-429

参考文献采用顺序编码制,按论文正文所引用文献出现的先后顺序连续编码。

\section{附录}

附录是作为报告、论文主体的补充项目,并不是必需的。

下列内容可以作为附录编于报告、论文后,也可以另编成册;

\begin{enumerate}
\item 为了整篇论文材料的完整,但编入正文又有损于编排的条理和逻辑性,这一材料
包括比正文更为详尽的信息、研究方法和技术更深入的叙述,建议可以阅读的参考文献
题录,对了解正文内容有用的补充信息等;
\item 由于篇幅过大或取材于复制品而不便于编入正文的材料;
\item 不便于编入正文的罕见珍贵资料;
\item 对一般读者并非必要阅读,但对本专业同行有参考价值的资料;
\item 某些重要的原始数据、数学推导、计算程序、框图、结构图、注释、统计表、计
算机打印输出件等。
\end{enumerate}

附录与正文连续编页码。

每一附录均另页起。

\section{结尾部分 (必要时)}

为了将论文迅速存储入电子计算机,可以提供有关的输入数据。可以编排分类索引、著者索
引、关键词索引等。

\chapter{“目次”和“目录”的区别}

通常我们都将书籍的章节列表称之为“目录”,但根据国家标准《GB/T 7713.1-2006: 学位论文编写规则》
\cite{gbt7713.1-2006}和《GB/T 13417-2009: 期刊目次表》\cite{gbt13417-2009},
以及《南京大学博士(硕士)学位论文编写格式规定(试行)》(参见附录\ref{chap:njureq})中的要求,
论文的章节列表其实应该称为“目次”。

那么“目次”究竟是什么意思?它和“目录”有何区别?

\section{百度百科的解释}

对于目录与目次的区别,百度百科上说明如下\cite{baidu2013muci}:

\begin{quotation}
Contents 目次

Table of Contents 目录

目次是目录的排序,目录是内容章节的具体名称。

目录专著始于刘向父子,到清代 《四库全书总目提要》里按经,史,子,集四类编目,每
一大类又分若干小类,类下分子目,大类前有总序,每一小类前有小序,子目后有案语,序
及案语是用来简述著作源流以及分类理由的。而每小类的后面还附有“存目” ,这存目就是
所谓的目次了 。目录又称为书目,是记录图书的书目名称著者,搜藏与流传情况,内容提
要,评价,真伪辨析等内容的。
\end{quotation}

不过百度百科的解释明显不清楚,且有错误。英语单词``content''在Merriam-Webster
在线版词典中作为名词的解释如下:
\begin{quotation}
a: something contained - usually used in plural;\\
b: the topics or matter treated in a written work;\\
c: the principal substance (as written matter, illustrations, 
or music) offered by a World Wide Web site.
\end{quotation}

显然后两个含义都是第一个含义的引申,即``content''作为名词其本意应该是指“内部包含
之物(something contained)”,后引申为文章的“内容”。而``table of contents''
则有“内容列表”的含义,和“目录”或“目次”相关。

\section{程千帆先生的意见}

程千帆先生反对用“目录”这个词。他说:“我写书时,对于底下的篇目我不用目录两个字的,
因为目是目,录是录,我总是写作目次,写篇目也可以,无论如何不能写目录。”\cite{cheng2008}

程先生的话与传统的目录学知识相关。刘向的《七略》是我国古代最早的全国综合性目录,
虽然今已失传,但由于《汉书·艺文志》是根据《七略》编写的,因此仍然可以知道其大概
体例。

\section{“目录”一词的由来}

《汉书·艺文志》中记载道:

\begin{quotation}
昔仲尼没而微言绝,七十子丧而大义乖。故《春秋》分为五,《诗》分为四,《易》有数家
之传。战国从衡,真伪分争,诸子之言纷然殽乱。至秦患之,乃燔灭文章,以愚黔首。汉兴,
改秦之败,大收篇籍,广开献书之路。迄孝武世,书缺简脱,礼坏乐崩,圣上喟然而称曰:
“朕甚闵焉!”于是建藏书之策,置写书之官,下及诸子传说,皆充秘府。至成帝时,以书颇
散亡,使谒者陈农求遗书于天下。诏光禄大夫刘向校经传诸子诗赋,步兵校尉任宏校兵书,
太史令尹咸校数术,侍医李柱国校方技。每一书已,向辄条其篇目,撮其指意,录而奏之。
会向卒,哀帝复使向子侍中奉车都尉歆卒父业。歆于是总群书而奏其《七略》,故有《辑略》,
有《六艺略》,有《诸子略》,有《诗赋略》,有《兵书略》,有《术数略》,有
《方技略》。今删其要,以备篇辑。
\end{quotation}

其中所说的“每一书已,向辄条其篇目,撮其指意,录而奏之”便是“目录”一词最初的用意了。
即“目”指“条其篇目”,“录”指“条其篇目,撮其指意,录而奏之”。这里采用的是余嘉锡先生在
《目录学发微·何谓目录》中的意见,也就是说“目”单指篇目,而“录”是包含篇目与内容大
意两部分内容的。而后由于袭用的缘故,“录”反而隶属在“目”之下了,于是有篇目而无内容
大意的也可称之为“目录”。再往后,只有书名而无篇目的也可称为“目录”。

《昭明文选·任彦开为范始兴求立太宰碑表(李善注)》引刘欲《七略》称:“《尚书》有青丝
编目录”,可知刘向校书,即已使用“口录”一词。班因《汉书·叙传》中,亦有“爱着目录,略序
洪烈,述艺文志第十”之句,表明早在汉代,“目录”二字已作为一个名词而被加以使用。

\section{“目录”一词的含义}

目录:是指著录一批相关文献,并按照一定次序编排而成的揭示与报道文献的工具。它是联系
文献与用户的桥梁和纽带。是书籍文章的缩影。

目录是目和录的总称。“目”指篇名或书名,“录”是对“目”的说明和编次。前人把“目”与“录”
编在一起,谓之“目录”。

\section{“目录”一词的其他含义}

在现代计算机技术术语中,常常把文件系统中的“directory”称为“目录”,表示
在文件系统树中的非叶子节点。这种用法给“目录”一词带来了新的含义。

\section{结论}

根据“目录”一词的由来,及程千帆先生的意见,我们决定遵循国家标准
《GB/T 7713.1-2006: 学位论文编写规则》\cite{gbt7713.1-2006}和
《GB/T 13417-2009: 期刊目次表》\cite{gbt13417-2009},以及
附录\ref{chap:njureq}中的要求,将``table of contents''称为
“目次”而非“目录”。



\chapter{姓名的格式}\label{chap:names}

\section{中国人的中文姓名}

\section{中国人的英文姓名}

\section{外国人的外文姓名}

\section{外国人名的中文译名}


%%%%%%%%%%%%%%%%%%%%%%%%%%%%%%%%%%%%%%%%%%%%%%%%%%%%%%%%%%%%%%%%%%%%%%%%%%%%%%%
% 论文附件
\backmatter
% 令所有未被引用的参考文献也出现在参考文献列表中
\nocite{*}
% 参考文献目录
\bibliography{njuthesis-manual}
% 作者简历与科研成果页,应放在参考文献之后
%%%%%%%%%%%%%%%%%%%%%%%%%%%%%%%%%%%%%%%%%%%%%%%%%%%%%%%%%%%%%%%%%%%%%%%%%%%%%%%
%%  
%% 文档类 NJU-Thesis 用户手册
%%
%% 作者:胡海星,starfish (at) gmail (dot) com
%% 项目主页: https://github.com/Haixing-Hu/nju-thesis
%%
%% This file may be distributed and/or modified under the conditions of the
%% LaTeX Project Public License, either version 1.2 of this license or (at your
%% option) any later version. The latest version of this license is in:
%%
%% http://www.latex-project.org/lppl.txt
%%
%% and version 1.2 or later is part of all distributions of LaTeX version
%% 1999/12/01 or later.
%%
%%%%%%%%%%%%%%%%%%%%%%%%%%%%%%%%%%%%%%%%%%%%%%%%%%%%%%%%%%%%%%%%%%%%%%%%%%%%%%%

\begin{resume}


\end{resume}

% 生成《学位论文出版授权书》页面,应放在最后一页
\makelicense
%%%%%%%%%%%%%%%%%%%%%%%%%%%%%%%%%%%%%%%%%%%%%%%%%%%%%%%%%%%%%%%%%%%%%%%%%%%%%%%
\end{document}
